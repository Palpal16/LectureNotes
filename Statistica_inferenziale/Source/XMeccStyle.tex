\usepackage{hyperref}
\usepackage[utf8]{inputenc}
\usepackage{amsfonts}
\usepackage{amsmath}
\usepackage{amssymb}
\usepackage{dsfont} % for using \mathds{1} characteristic function
\usepackage{bbm}
\usepackage{relsize}
\usepackage{pgfplots}
\pgfplotsset{compat=1.18}
\usepackage{scalerel}
\usepackage{wrapfig}
\usepackage[T1]{fontenc}       % change font encoding to T1
\usepackage[framed,numbered]{matlab-prettifier}


\setlength{\parindent}{0cm} %toglie lo spazio dato dal del doppio invio

\renewcommand{\contentsname}{Indice}


\usepackage{amsthm}
\definecolor{Green}{RGB}{0,210,100}
\definecolor{Black}{RGB}{0,0,0}
\newtheoremstyle{DEFstyle} % Theorem style name
{0pt}% Space above
{0pt}% Space below
{\normalfont}% Body font
{}% Indent amount
{\bf\scshape}% Theorem head font --- {\small\bf}
{.\\}% Punctuation after theorem head
{0em}% Space after theorem head
{\small\thmname{#1}% Theorem text (e.g. Theorem 2.1)
%{\small\thmname{#1}% Theorem text (e.g. Theorem)
\thmnote{:\nobreakspace\normalfont\bfseries \nobreakspace#3}}% Optional theorem note



\newtheoremstyle{DIMstyle} % Theorem style name
{0pt}% Space above
{0pt}% Space below
{\normalfont}% Body font
{}% Indent amount
{\bf\scshape}% Theorem head font --- {\small\bf}
{\\}% Punctuation after theorem head
{0em}% Space after theorem head
{\small\thmname{#1}.% Theorem text (e.g. Theorem 2.1)
%{\small\thmname{#1}% Theorem text (e.g. Theorem)
\thmnote{\nobreakspace\normalfont\nobreakspace(#3)}}% Optional theorem note

\newtheoremstyle{RIPstyle} % Theorem style name
{0pt}% Space above
{0pt}% Space below
{\normalfont}% Body font
{}% Indent amount
{\bf\scshape}% Theorem head font --- {\small\bf}
{\\}% Punctuation after theorem head
{0em}% Space after theorem head
{\small\thmname{#1}:% Theorem text (e.g. Theorem 2.1)
%{\small\thmname{#1}% Theorem text (e.g. Theorem)
\thmnote{\nobreakspace\normalfont\nobreakspace#3}}% Optional theorem note



\theoremstyle{DEFstyle}
\newtheorem* {theoremT}{Definizione}
\newtheorem* {theoremT1}{Teorema}
\theoremstyle{DIMstyle}
\newtheorem* {theoremT2}{Dim}
\theoremstyle{RIPstyle}
\newtheorem* {theoremT3}{Ripasso}
\RequirePackage[framemethod=default]{mdframed} % Required for creating the theorem, definition, exercise and corollary boxes
% green box
\newmdenv[skipabove=7pt,
skipbelow=7pt,
rightline=false,
leftline=true,
topline=false,
bottomline=false,
linecolor=Green,
backgroundcolor=green!0,
innerleftmargin=5pt,
innerrightmargin=5pt,
innertopmargin=5pt,
leftmargin=0cm,
rightmargin=0cm,
linewidth=2pt,
innerbottommargin=5pt]{gBox}


\newmdenv[skipabove=7pt,
skipbelow=7pt,
rightline=false,
leftline=true,
topline=false,
bottomline=false,
linecolor=blue,
backgroundcolor=green!0,
innerleftmargin=5pt,
innerrightmargin=5pt,
innertopmargin=5pt,
leftmargin=0cm,
rightmargin=0cm,
linewidth=2pt,
innerbottommargin=5pt]{bBox}


\newmdenv[skipabove=7pt,
skipbelow=20pt,
rightline=false,
leftline=true,
topline=false,
bottomline=false,
linecolor=Black,
backgroundcolor=green!0,
innerleftmargin=5pt,
innerrightmargin=5pt,
innertopmargin=3pt,
leftmargin=0cm,
rightmargin=0cm,
linewidth=0.5pt,
innerbottommargin=5pt]{dimBox}


\newmdenv[skipabove=7pt,
skipbelow=20pt,
rightline=false,
leftline=true,
topline=false,
bottomline=false,
linecolor=Black,
backgroundcolor=green!0,
innerleftmargin=10pt,
innerrightmargin=5pt,
innertopmargin=3pt,
leftmargin=0cm,
rightmargin=0cm,
linewidth=0.5pt,
innerbottommargin=17pt]{ripBox}


\newenvironment{defi}{\begin{gBox}\begin{theoremT}}{\end{theoremT}\end{gBox}}
\newenvironment{teo}{\begin{bBox}\begin{theoremT1}}{\end{theoremT1}\end{bBox}}
\newenvironment{Dim}{\begin{dimBox}\begin{theoremT2}}{\phantom{}\hfill$\qed$\end{theoremT2} \end{dimBox} }
\newenvironment{RIP}{\begin{ripBox}\begin{theoremT3}}{\end{theoremT3}\end{ripBox}}


\title{Modelli e Metodi analitici per le EDP}
\author{Simone Paloschi}
\date{INGMTM \ \ A.A. 2022/2023}
\linespread{1.5}

\DeclareRobustCommand{\Chi}{{\mathpalette\irchi\relax}}
\newcommand{\irchi}[2]{\raisebox{\depth}{$#1\chi$}} % inner command, used by \Chi


\newcommand{\EE}{\mathbb E}
\newcommand{\NN}{\mathbb N}
\newcommand{\PP}{\mathbb P}
\newcommand{\RR}{\mathbb R}
\newcommand{\ZZ}{\mathbb Z}

\newcommand{\Ac}{\mathcal A}
\newcommand{\Bc}{\mathcal B}
\newcommand{\Cc}{\mathcal C}
\newcommand{\Ec}{\mathcal E}
\newcommand{\Fc}{\mathcal F}
\newcommand{\Gc}{\mathcal G}
\newcommand{\Hc}{\mathcal H}
\newcommand{\Lc}{\mathcal L}
\newcommand{\Nc}{\mathcal N}
\newcommand{\Pc}{\mathcal P}
\newcommand{\Uc}{\mathcal U}




