\usepackage{hyperref}
\usepackage[utf8]{inputenc}
\usepackage{amsfonts}
\usepackage{amsmath}
\usepackage{amssymb}
\usepackage{dsfont} % for using \mathds{1} characteristic function
\usepackage{tikz}
\usepackage{bbm}
\usepackage{relsize}
\usepackage{pgfplots}
\pgfplotsset{compat=1.18}
\usepackage{scalerel}
\usepackage{wrapfig}
\usepackage[T1]{fontenc}       % change font encoding to T1
\usepackage[framed,numbered]{matlab-prettifier}


%toglie lo spazio dato dal del doppio invio
\setlength{\parindent}{0cm} 

%distanza dal numero di pagina
\setlength{\footskip}{1.3cm}


\usepackage[
	left=2.5cm, % inner
	right=2.5cm, % outer
	top=2.3cm,
	bottom=2.8cm,
	%showframe,
	]{geometry}



\graphicspath{ {./images/} }


\renewcommand{\contentsname}{Indice}



\title{Modelli e Metodi dell'inferenzza Statistica}
\author{Simone Paloschi}
\date{INGMTM \ \ A.A. 2022/2023}
\linespread{1.5}

\DeclareRobustCommand{\Chi}{{\mathpalette\irchi\relax}}
\newcommand{\irchi}[2]{\raisebox{\depth}{$#1\chi$}} % inner command, used by \Chi


\newcommand{\EE}{\mathbb E}
\newcommand{\NN}{\mathbb N}
\newcommand{\PP}{\mathbb P}
\newcommand{\R}{\mathbb R}
\newcommand{\ZZ}{\mathbb Z}

\newcommand{\Ac}{\mathcal A}
\newcommand{\Bc}{\mathcal B}
\newcommand{\Cc}{\mathcal C}
\newcommand{\Ec}{\mathcal E}
\newcommand{\Fc}{\mathcal F}
\newcommand{\Gc}{\mathcal G}
\newcommand{\Hc}{\mathcal H}
\newcommand{\Lc}{\mathcal L}
\newcommand{\Nc}{\mathcal N}
\newcommand{\Pc}{\mathcal P}
\newcommand{\Uc}{\mathcal U}
\newcommand{\Vc}{\mathcal V}


\usepackage{esvect}

%%%%% shortcuts

\newcommand{\eps}{\varepsilon}




%displaystyle frazione e integrale
\renewcommand{\frac}{\dfrac}

\let \INT \int 
\renewcommand{\int}{\displaystyle\INT}


\newcommand{\ind}{\perp \!\!\! \perp} % indipendenza

%displaystyle sommatoria e prod
\newcommand{\Sum}[2]{\sum\limits_{#1}^{#2}} 

\newcommand{\Prod}[2]{\prod\limits_{#1}^{#2}}



%displaystyle parentesi

\renewcommand{\O}{\left(}
\newcommand{\C}{\right)}

\newcommand{\OO}{\left[}
\newcommand{\CC}{\right]}

\newcommand{\OOO}{\left\{}
\newcommand{\CCC}{\right\}}


%matrice indicatrice
\newcommand{\II}{\mathds{1}}

%insieme vuoto
\renewcommand{\empty}{\emptyset}


\renewcommand{\hat}[1]{\widehat{#1}}
\renewcommand{\tilde}[1]{\widetilde{#1}}


 \newcommand{\Asterisk}{\mathop{\scalebox{1.5}{\raisebox{-0.2ex}{$\ast$}}}} %for big * 











%%%%%%%%%%%%%%%%%%%%%%%%%%%%%%
%AMBIENTI TEOREMI
%%%%%%%%%%%%%%%%%%%%%%%%%%%%%%



\usepackage{amsthm}
\definecolor{Green}{RGB}{0,210,100}
\definecolor{Black}{RGB}{0,0,0}
\newtheoremstyle{DEFstyle} % Theorem style name
{0pt}% Space above
{0pt}% Space below
{\normalfont}% Body font
{}% Indent amount
{\bf\scshape}% Theorem head font --- {\small\bf}
{.\\}% Punctuation after theorem head
{0em}% Space after theorem head
{\small\thmname{#1}% Theorem text (e.g. Theorem 2.1)
%{\small\thmname{#1}% Theorem text (e.g. Theorem)
\thmnote{:\nobreakspace\normalfont\bfseries \nobreakspace#3}}% Optional theorem note



\newtheoremstyle{DIMstyle} % Theorem style name
{0pt}% Space above
{0pt}% Space below
{\normalfont}% Body font
{}% Indent amount
{\bf\scshape}% Theorem head font --- {\small\bf}
{\\}% Punctuation after theorem head
{0em}% Space after theorem head
{\small\thmname{#1}.% Theorem text (e.g. Theorem 2.1)
%{\small\thmname{#1}% Theorem text (e.g. Theorem)
\thmnote{\nobreakspace\normalfont\nobreakspace( #3 )}}% Optional theorem note

\newtheoremstyle{RIPstyle} % Theorem style name
{0pt}% Space above
{0pt}% Space below
{\normalfont}% Body font
{}% Indent amount
{\bf\scshape}% Theorem head font --- {\small\bf}
{\\}% Punctuation after theorem head
{0em}% Space after theorem head
{\small\thmname{#1}:% Theorem text (e.g. Theorem 2.1)
%{\small\thmname{#1}% Theorem text (e.g. Theorem)
\thmnote{\nobreakspace\normalfont\nobreakspace#3}}% Optional theorem note



\theoremstyle{DEFstyle}
\newtheorem* {theoremT}{Definizione}
\newtheorem* {theoremT1}{Teorema}
\theoremstyle{DIMstyle}
\newtheorem* {theoremT2}{Dim}
\theoremstyle{RIPstyle}
\newtheorem* {theoremT3}{Ripasso}
\RequirePackage[framemethod=default]{mdframed} % Required for creating the theorem, definition, exercise and corollary boxes
% green box
\newmdenv[skipabove=7pt,
skipbelow=7pt,
rightline=false,
leftline=true,
topline=false,
bottomline=false,
linecolor=Green,
backgroundcolor=green!0,
innerleftmargin=5pt,
innerrightmargin=5pt,
innertopmargin=5pt,
leftmargin=0cm,
rightmargin=0cm,
linewidth=2pt,
innerbottommargin=5pt]{gBox}


\newmdenv[skipabove=7pt,
skipbelow=7pt,
rightline=false,
leftline=true,
topline=false,
bottomline=false,
linecolor=blue,
backgroundcolor=green!0,
innerleftmargin=5pt,
innerrightmargin=5pt,
innertopmargin=5pt,
leftmargin=0cm,
rightmargin=0cm,
linewidth=2pt,
innerbottommargin=5pt]{bBox}


\newmdenv[skipabove=7pt,
skipbelow=20pt,
rightline=false,
leftline=true,
topline=false,
bottomline=false,
linecolor=Black,
backgroundcolor=green!0,
innerleftmargin=5pt,
innerrightmargin=5pt,
innertopmargin=3pt,
leftmargin=0cm,
rightmargin=0cm,
linewidth=0.5pt,
innerbottommargin=5pt]{dimBox}


\newmdenv[skipabove=7pt,
skipbelow=20pt,
rightline=false,
leftline=true,
topline=false,
bottomline=false,
linecolor=Black,
backgroundcolor=green!0,
innerleftmargin=10pt,
innerrightmargin=5pt,
innertopmargin=3pt,
leftmargin=0cm,
rightmargin=0cm,
linewidth=0.5pt,
innerbottommargin=17pt]{ripBox}


\newenvironment{defi}{\begin{gBox}\begin{theoremT}}{\end{theoremT}\end{gBox}}
\newenvironment{teo}{\begin{bBox}\begin{theoremT1}}{\end{theoremT1}\end{bBox}}
\newenvironment{Dim}{\begin{dimBox}\begin{theoremT2}}{\phantom{}\hfill$\qed$\end{theoremT2} \end{dimBox} }
\newenvironment{RIP}{\begin{ripBox}\begin{theoremT3}}{\end{theoremT3}\end{ripBox}}





%%%%%%%%%%%%%%%%%%%%%%%%%%%%%%
%FIGURA CENTRALE
%%%%%%%%%%%%%%%%%%%%%%%%%%%%%%

\usepackage{caption}\captionsetup{belowskip=12pt,aboveskip=4pt}
\usepackage{placeins} % The placeins package gives the command \FloatBarrier, which will make sure any floats will be put in before this point
\usepackage{flafter}  % The flafter package ensures that floats don't appear until after they appear in the code.

\usepackage{graphicx}
\usepackage{float}

%chiamata:   \fg[ "caption" ]{0.5}{figure.jpeg}
%senza caption non viene \fg{0.5}{figure.jpeg}
\newcommand{\fg}[3][\relax]{%
  \begin{figure}[H]%[htp]%
    \centering
    \captionsetup{width=0.7\textwidth}
      \includegraphics[width = #2\textwidth]{#3}%
      \ifx\relax#1\else\caption{#1}\fi
      \label{#3}
  \end{figure}%
  \FloatBarrier%
}



\usepackage{cancel} %per usare \cancel e \bcancel

\newcommand{\skipp}{\smallskip\smallskip}




%per mettere la graffa sotto al testo
\newcommand{\UB}[2]{\underset{#1}{\underbrace{#2}}}