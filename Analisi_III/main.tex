\documentclass{article}
\usepackage{hyperref}
\usepackage[utf8]{inputenc}
\usepackage{amsfonts}
\usepackage{amsmath}
\usepackage{amssymb}
\usepackage{dsfont} % for using \mathds{1} characteristic function
\usepackage{tikz}
\usepackage{bbm}
\usepackage{relsize}
\usepackage{pgfplots}
\usepackage{scalerel}
\usepackage{wrapfig}
\usepackage[T1]{fontenc}       % change font encoding to T1
\usepackage[framed,numbered]{matlab-prettifier}
\pgfplotsset{width=10cm,compat=1.9}
\addtolength{\oddsidemargin}{-.875in}
	\addtolength{\evensidemargin}{-.875in}
	\addtolength{\textwidth}{1.75in}

	\addtolength{\topmargin}{-.875in}
	\addtolength{\textheight}{1.75in}

\usepackage{amsthm}
\definecolor{Green}{RGB}{0,210,100}
\newtheoremstyle{unnumbered} % Theorem style name
{0pt}% Space above
{0pt}% Space below
{\normalfont}% Body font
{}% Indent amount
{\bf\scshape}% Theorem head font --- {\small\bf}
{.\;}% Punctuation after theorem head
{0.2em}% Space after theorem head
{\small\thmname{#1}\thmnumber{\@ifnotempty{#1}{}\@upn{#2}}% Theorem text (e.g. Theorem 2.1)
%{\small\thmname{#1}% Theorem text (e.g. Theorem)
\thmnote{\nobreakspace\the\thm@notefont\normalfont\bfseries---\nobreakspace#3}}% Optional theorem note

\newtheoremstyle{unnumbered1} % Theorem style name
{0pt}% Space above
{0pt}% Space below
{\normalfont}% Body font
{}% Indent amount
{\bf\scshape}% Theorem head font --- {\small\bf}
{\!\!\;}% Punctuation after theorem head
{0em}% Space after theorem head
{\small\thmname{#1}\thmnumber{\@ifnotempty{#1}{}\@upn{#2}}% Theorem text (e.g. Theorem 2.1)
%{\small\thmname{#1}% Theorem text (e.g. Theorem)
\thmnote{\nobreakspace\the\thm@notefont\normalfont\bfseries---\nobreakspace#3}}% Optional theorem note

\theoremstyle{unnumbered}
\newtheorem* {theoremT}{Def}
\theoremstyle{unnumbered1}
\newtheorem* {theoremT1}{}
\RequirePackage[framemethod=default]{mdframed} % Required for creating the theorem, definition, exercise and corollary boxes
% green box
\newmdenv[skipabove=7pt,
skipbelow=7pt,
rightline=false,
leftline=true,
topline=false,
bottomline=false,
linecolor=Green,
backgroundcolor=green!0,
innerleftmargin=5pt,
innerrightmargin=5pt,
innertopmargin=5pt,
leftmargin=0cm,
rightmargin=0cm,
linewidth=2pt,
innerbottommargin=5pt]{gBox}


\newenvironment{defi}{\begin{gBox}\begin{theoremT}}{\end{theoremT}\end{gBox}}
\newenvironment{Ndefi}{\begin{gBox}\begin{theoremT1}}{\end{theoremT1}\end{gBox}}


\title{Analisi III}
\author{Simone Paloschi}
\date{INGMTM \ \ A.A. 2022/2023}
%\linespread{1.5}

\DeclareRobustCommand{\Chi}{{\mathpalette\irchi\relax}}
\newcommand{\irchi}[2]{\raisebox{\depth}{$#1\chi$}} % inner command, used by \Chi

\renewcommand{\contentsname}{Indice}
\newcommand{\eps}{\varepsilon}
\renewcommand{\phi}{\varphi}


%
%
%
%
%SOLO PER QUESTO
\linespread{1.2}
\renewcommand{\hat}{\widehat}
%%%%%%%%%%%%%


\begin{document}




\vspace*{\fill}
\begin{center}
	{\Huge \textsc{Analisi Matematica III}}\\
	\vspace*{1.4cm}
	{\large {Dalle lezioni del Prof. Maurizio Grasselli}}\\
	\vspace*{0.2cm}
	{\large per il corso di Ingegneria Matematica}\\
	\vspace*{1cm}
	{\large {Dispense di Simone Paloschi}}\\
	\vspace*{1cm}
	Politecnico di Milano\\
	\vspace*{0.2cm}
	A.A. 2022/2023
\end{center}
\vspace{7cm}
\vspace*{\fill}

\newpage

\begingroup
  \hypersetup{hidelinks}
  \tableofcontents
\endgroup


\pagebreak


\section{Funzioni a variabile complessa}

\subsection{
Insieme numeri complessi}
%Lez1
Oss. Il campo ($\mathbb{C}$,+,$\cdot$) non può essere ordinato \\
In $\mathbb{C}$ vale il teorema fondamentale dell'algebra: un polinomio di grado n ha n radici\\ \\
%
I numeri complessi possono essere scritti: $a +ib \ \ \rho(cos(\eta)+isin(\eta)) \ \ \rho e^{i\eta}$\\ \\
%
Topologia:\\
\ • \ \textbf{distanza} := |z-w| \ è  simmetrica e positiva \ ($d=0 \Leftrightarrow z=w$)\\
%
\ • \ \textbf{Disco}  (aperto) \ \ $R>0 \ e \ z_0 \in \mathbb{C} \ \ allora \ B_R(z_0) : = { z\in \mathbb{C} : |z-z_0|<R } \ \ \ \text{ e chiuso con } \le$\\
%
\ • \ $A\subseteq\mathbb{C}$ si dice \textbf{aperto} se \ $\forall z_0 \in A \ \ \exists R>0 \ t.c. \ B_R(z_0) \subset A$\\
%
\ • \ Punto di \textbf{accumulazione} se $\ \forall R>0 \ B_R(z_0)  \text{ contiene almeno uno } z \in E \ (con \ z \neq z_0)$\\
%
\ • \ $E\subseteq\mathbb{C}$ \textbf{chiuso} se contiene tutti i suoi punti di accumulazione \\
\ • \ Una succesione $[z_n]_{n\in\mathbb{N} } \subset \mathbb{C} $ converge a $ z \in \mathbb{C} \ se \ \forall \varepsilon >0 \ \exists n_0 \in \mathbb{N} \ t.c. \ |z_n-z| < \varepsilon, \forall n>n_0$ \\




\subsection{Funzioni Complesse di una Variabile Complessa}
f: $E\subseteq\mathbb{C}\rightarrow\mathbb{C}$,  f(z)=Ref(z)  +  i Imf(z)
%
\begin{defi}
f è \textbf{continua} in $z_0 	\in A \ \ se \ \forall \varepsilon >0,\ \exists \delta >0 \ t.c.\ |f(z)-f(z_0)|<\varepsilon \ \forall z \in B_{\delta} (z_0) \cap  A$
\end{defi}
%
Oss. f(z) = u(x+iy) + i v(x+iy) (u e v funzioni) \\ data f possiamo definire in modo univoco $(u,v): E \subset\mathbb{R}^2\rightarrow\mathbb{R}^2 $ \ e viceversa
%
\begin{defi}
Sia $f:A\subseteq\mathbb{C}\rightarrow\mathbb{C}$ \ diremo che f è \textbf{derivabile} in $z_0\in A$ se esiste finito  \ $lim_{h\rightarrow 0} \frac{f(z_0 + h)-f(z_0)}{h} = f'(z_0)$
\end{defi}
%
Oss.  Ovvero $\forall\varepsilon >0 , \ \exists \delta>0 \ t.c. \ \lvert \frac{f(z_0+h)-f(z_0)}{h} - f'(z_0) \rvert <\varepsilon \ \ \ \  \forall h \neq 0 \ t.c. \ |h|<\delta \ e \ z_0+h \in A $\\ \\
%
Oss. Derivabilità equivale a  \textbf{differenziabilità}:\\ $\exists \alpha \in \mathbb{C} \ t.c. \ f(z_0 +h) -f(z_0) = \alpha h + o(h) \ \ per \ h \rightarrow 0 \ \ con\ \alpha=f'(z_0)$\\ \\
%
%
Teo. Condizioni di Cauchy-Riemann \ \ \ DIM\\
f è derivabile in $z_0 \Longleftrightarrow$ u,v sono diffferenziabili in $(x_0,y_0)$, \ $\frac{\partial u}{\partial x}(x_0,y_0) = \frac{\partial v}{\partial y}(x_0,y_0) \ e \ \ \frac{\partial u}{\partial y}(x_0,y_0) = - \frac{\partial v}{\partial x}(x_0,y_0)$\\ 
%
%
\begin{defi}$f:A\subseteq\mathbb{C}\rightarrow\mathbb{C}$ è \textbf{Olomorfa} in A se è derivabile $\forall z\in A$
\end{defi}
\phantom{}\\
%
%
Oss. $f:A\subseteq\mathbb{C}\rightarrow\mathbb{C}$ è derivabile in $z_0=x_0+iy_0 \ \in A$, allora\\
$f'(z0) = \frac{\partial u}{\partial x}(x_0,y_0) + i\frac{\partial v}{\partial x}(x_0,y_0) = (C.R.) = \frac{\partial v}{\partial y}(x_0,y_0) -i \frac{\partial u}{\partial y}(x_0,y_0) \implies |f'(z0)|^2 = u_x^2+v_x^2 = v_y^2+u_y^2$ \\
%
%
Oss. $F:A\subseteq\mathbb{R}^2\rightarrow\mathbb{R}^2 \ \ F(x,y)=(u(x,y),v(x,y))$ \ \ \ $J_F \ (x,y)=|f'(z_0)|
\begin{bmatrix}
\frac{u_x}{|f'(z_0)|} \ \frac{v_x}{|f'(z_0)|} \\ \frac{u_y}{|f'(z_0)|} \ \frac{u_y}{|f'(z_0)|}
\end{bmatrix}
 =|f'(z_0)| \theta(z_0)$\\
%
%Lez2
%
\begin{defi}
Una trasformazione di un aperto si dice \textbf{conforme} se conserva gli angoli tra coppie di curve (regolari)
\end{defi}
%
Teo. f olomorfa in A con $f'(z)\neq 0 \ \forall z \in A$ \ \ allora f è una trasformazione conforme \\ \\
%
Oss. Vogliamo mostrare che le funzioni olomorfe sono serie di potenze %ci vorrebbe \\


\subsection{Serie Di Potenze nel campo complesso}
%
\begin{defi}Una serie di potenze di centro $z_0\in\mathbb{C}$ e coefficienti $a_n\in\mathbb{C}$ è la successione $\{\sum_{k=0}^n a_k(z-z_0)^k  \}_{n\in\mathbb{N}}$\\
che può essere chiamata con il simbolo \ $\sum_{n=0}^{\infty} a_n(z-z_0)^n$
\end{defi}
%
Oss.  Puoi sostituire la serie alla successione se sai che converge\\ \\
%
%
Teo.  Raggio di convergenza delle serie di potenze nel campo complesso \ \ \ DIM\\
Sia $\alpha=\overline{lim}_{n\rightarrow +\infty} |a_n|^{\frac{1}{n}} \in [0,+\infty]$ \ allora posto $R=\frac{1}{\alpha}$, avremo:\\
i) la serie $\sum_{n\in\mathbb{N}}a_n(z-z_0)^n$ converge assolutamente $\ \forall z : |z-z_0|<R$\\
ii) la serie non converge $\ \forall z : |z-z_0|>R$
%
%
\begin{defi}
Data la serie $\sum_{n\in\mathbb{N}}a_n(z-z_0)^n$ con raggio di convergenza R\\
Il \textbf{cerchio di convergenza} è $\Gamma_R(z_0)=\{z\in\mathbb{C} : |z-z_0|<R \}$\\
Il \textbf{disco} di convergenza è $\mathcal{D}_R(z_0)=\{z\in\mathbb{C}^* : |z-z_0|\le R \} \ \ \ con \ \mathbb{C}^*: \mathbb{C} \ \cup \{+\infty\}$
\end{defi}
%
%
Oss.  $\forall R'<R$ la serie converge totalmente in $|z-z_0|<R'$\\ \\
%
Cor. Una serie di potenze converge totalmente e uniformemente in ogni E t.c. $\overline{E}\subset\Gamma_R(z_0)$\\ \\
%
%
Teo. Una serie di potenze ha come somma una funzione continua in $|z-z_0|\le R $ \\
scriveremo \ $f(z)= \sum_{n=0}^{\infty} a_n(z-z_0)^n$ \\ \\
%
%
Prop. $ \sum_{n=0}^{\infty} a_n(z-z_0)^n \ \text{ e } \  \sum_{n=1}^{\infty} na_n(z-z_0)^{n-1}$ (Serie derivata termine a termine)\\  hanno lo stesso raggio di convergenza\\ \\
%
%
Teo. Derivabilità delle serie di potenze \ \ \  DIM \\
La somma f(z) di una serie di potenze è olomorfa in $|z-z_0|\le R $ e risulta $f'(z)= \sum_{n=1}^{\infty} na_n(z-z_0)^{n-1}$\\ \\
%
%
Cor.   La somma ha derivate di ogni ordine continue in $|z-z_0|<R$ che si ottengono derivando termine a termine: \ \
$f^{(k)}(z)=\sum_{n=k}^{+\infty} \frac{n!}{(n-k)!}a_n(z-z_0)^{n-k} \ \ \text{ e in particolare } \ f^{(k)}(z_0)=k!\ a_k$\\ \\ \\
%
%
\textbf{Estensione delle trascendenti elementari} \ \ \ (sin, cos, exp…)\\ \\
%
Teo.  Sia $f:(a,b)\subseteq\mathbb{C}$  derivabile in (a,b).\\ Se esiste una funzione olomorfa in A $\supset$ (a,b) \ t.c. $F|_{(a,b)} =f$ allora F è unica\\ \\
%
Cor. Sono funzioni olomorfe le seguenti:\\
$\sum_{n=0}^Na_nz^n \ \ \ \ \ \sum_{n=0}^{+\infty}\frac{z^n}{n!} :=e^z \ \ \ \ \ \sum_{n=0}^{+\infty}(-1)^n\frac{z^{2n+1}}{(2n+1)!} :=sin (z) \ \ \ \ \ \sum_{n=0}^{+\infty}(-1)^n\frac{z^{2n}}{(2n)!} :=cos (z)$\\ \\
%
%Lez3
%
Cor. Relazioni fondamentali che continuano a valere:\\
$sin^2z+cos^2z=1 \ \ \ \ sinz=\frac{e^{iz}-e^{-iz}}{2i} \ \ \ \ cosz=\frac{e^{iz}+e^{-iz}}{2} \ \ \ e^{z_1+z_2}=e^{z_1}\cdot e^{z_2}$\\
%
Oss. Ma per esempio non è più vera: $|sinz|\le 1$\\
\newpage




\subsection{Cammini e circuiti}
%
\begin{defi}$r:[a,b]\rightarrow \mathbb{C}$ è $\mathbf{C^1}$ \textbf{a tratti} se è $C^0$ ed $\exists \{t_0...t_k\}\ t.c.\ \ \ t_0=a, t_k=b, t_{j-1}<t_j\ e\ \ r|_{[t_{}j-1,t_j]} \in C^1$
\end{defi}
%
\begin{defi}$r_1,r_2 \in \widetilde{C}^1$ sono \textbf{equivalenti} se $\exists\phi:D_1\rightarrow D_2$ biettiva, strettamente crescente\\ con $D_i$ i rispettivi domini t.c. \ \ \ i) $\phi,\phi^{-1}\in \widetilde{C}^1$ \ \ \ ii) $r_1=r_2(\phi)$
\end{defi}
%
\begin{defi}Sia $C=(\gamma,{r})$ un \textbf{cammino} di $\mathbb{C}$, allora: \\
i) C è chiuso se $r(a)=r(b)$ \ \ (allora C è un \textbf{circuito})\\
ii) -C indica il cammino \textbf{inverso}, ovvero $\widetilde{r}(t)=r(a+b-t)$\\
iii) se $C\in(a,b)$, allora è la \textbf{somma} dei cammini parametrizzati da $r_1=r|_{[a,l]} \ e \ r_2=r|_{[l,b]}$: \ \ $C=C_1+C_2$\\
iv) C è un cammino \textbf{semplice} se non ha intersezioni
\end{defi}
%
\begin{defi}$A\subseteq\mathbb{C}$ è \textbf{connesso} se $\nexists A_1,A_2\subset A$ non vuoti, digiunti t.c. $A_1\cup A_2=A$ % \ A_1\cap\overline{A}_2\neq\emptyset \ e \ \overline{A}_1\cap A_2\neq\emptyset$
\end{defi}
%
\begin{defi}Siano $ C_1,C_2$ circuiti in $A\subseteq\mathbb{C}$, \textbf{A-omotopia} è una funzione $H:[a,b]\times[0,1]\rightarrow A$ continua t.c. \\ i) $\forall \lambda \in [0,1] \ H(\cdot,\lambda)$ è la parametrizzazione di un circuito di A \\ ii) $H(\cdot,0) \ e \ H(\cdot,1) $ sono parametrizzazioni di $C_1 \ e \ C_2$ che si diranno A-omotopi
\end{defi}
%
\begin{defi}A è \textbf{semplicemente connesso} se ogni circuito in A è A-omotopo ad un punto di A
\end{defi}
%
\begin{defi}A è \textbf{stellato} rispetto a $z_0\in A$ se ogni segmento di $z_0$ è incluso in A \ \ (Stellato $\implies$ sempl connesso)
\end{defi}


\subsection{Integrale di funzioni complesse di variabile complessa}
%
\begin{defi}
Siano $C=(\gamma,\{r\})$ un cammino di $\mathbb{C}$ e $f:\gamma\rightarrow\mathbb{C}$ continua, avremo \ \ $\int_C f(z)dz :=\int_a^b f(r(t))\cdot r'(t)dt $
\end{defi}
%
Proprietà principali:\\
i)\ \ lunghezza $L_{\gamma}=\int_a^b|r'(t)|dt$ \hspace{0.35in} ii)\ $\int_{-C}f(z)dz=-\int_Cf(z)dz$\\
iii) $\int_{C_1+C_2} f(z)dz =\int_{C}f(z)dz$\hspace{0.35in}
iv) $|\int_C f(z)dz| \le sup_{z\in\gamma} |f(z)|\cdot L_{\gamma}$\\ \\
%
Possiamo separare l'integrale: \ $\int_C f(z)dz = ... = \int_C (u(x,y)dx - v(x,y)dy) +i\int_C (v(x,y)dx + u(x,y)dy)$\\
%
%CAUCHY
%
\begin{defi}
Sia f continua con (u,v), allora $udx-vdy, vdx+udy$ si dicono \textbf{forme differenziali} associate a f
\end{defi}
%
Teo. f è olomorfa se le forme diff associate sono chiuse (differenziabili e $C^1$)\\
%
%Lez4
%
%dim. forme differenziali chiuse $\Longleftrightarrow \ rot\underline{F}=0 \Longleftrightarrow \ \partial xF_2 = \partial y F_1\ (\text{essendo } F_3=0) \ \Longleftrightarrow $ f olomorfa \\
%
%Oss: Il cammino $\int_a^b f(r(t))\cdot r'(t)dt $ è il limite di somme di C.-R. $\ \sum_{j=1}^n f(r(t_j^*))(r(t_j)-r(t_{j-1}))$\\
%
\begin{defi}
$F:A\subseteq\mathbb{C}\rightarrow\mathbb{C}$ è una \textbf{primitiva} di $f:A\subseteq\mathbb{C}\rightarrow\mathbb{C}$ se  è olomorfa e $F'(z)=f(z) \ \forall z \in A$
\end{defi}
%
Oss: Considerando F con (U,V), allora $U_x=V_y=u, \ V_x=-U_y=v \ $\\\ ovvero $\nabla U=(u,-v) \ \nabla V=(v,u)$, perciò le forme diff di f sono esatte (hanno primitiva)\\
%
Teo. f continua ha primitiva in A $\Longleftrightarrow$ le sue f.d.ass. sono esatte\\ \\
%
Teo. forma diff esatta e $C^1 \implies$ f.d. chiusa \\
Ipotesi: d'ora in poi assumeremo $u,v\in C^1$\\ \\
%
Teo. f continua ha primitiva in A $\implies$ f è olomorfa in A (essendo chiusa)\\ \\
%
Teo. f olomorfa su A semplicemente connesso $\Longleftrightarrow$ f ha primitiva in A\\
%
Cor. f continua e olomorfa in A $\Longleftrightarrow \ \forall z \in A \ \exists$ un disco $B_R(z)\subset A$ t.c. f ha primitiva in $B_R$ \\ \\
%
Teo. f continua con una primitiva in A $\implies \int_{C}f(z)dz=0 \ \forall \text{ circuito }C\subset A$\\
%
Cor. fissato $z_0\in A, \ F(z)=\int_{C(z_0,z)}f(w)dw$ è una primitiva di f, con C un cammino con estemi $z_0 \ e \ z$ \\ \\
%
Teo. Cauchy \ \ \ DIM\\
Sia $f:A\subseteq\mathbb{C}\rightarrow\mathbb{C}$ olomorfa. Se A è semplicemente connesso, allora $\int_{C}f(z)dz=0 \ \forall \text{ circuito }C\subset A$\\ \\
%
Teo. Morera\\
Se f è t.c. $\int_{C}f(z)dz=0 \ \forall \text{ circuito }C\subset A$ allora f è olomorfa \\





\subsection{Analicità funzioni olomorfe}
%
\begin{defi}
f è \textbf{analitica} se si può scrivere localmente in serie di potenze, ovvero:\\
$\forall z \in A \ \exists\delta>0 \ t.c. \ B_{\delta}(z)\subset A$ \
ed $\exists$ una serie di potenze di centro $z$ che converge a $f(z+h) \ \ \forall h\in B_{\delta}$
\end{defi}
%
Oss. Nel caso reale esistono funzioni $C^{\infty}$ non analitiche \\ \\
%
Teo. Se f è analitica allora è olomorfa\\ \\
%
Teo. Formula di Cauchy \ \ \ DIM\\
Se $f:A\subseteq\mathbb{C}\rightarrow\mathbb{C}$ olomorfa in A, allora $\forall z \in B_r(z_0)\subset A \ \ \ f(z)= \frac{1}{2\pi i} \int_{C_r(z_0)} \frac{f(w)}{w-z}dw $ \ \ con $C_r(z_0) = \partial B_r(z_0)$\\
%
Oss. I valori di f nel disco dipendono solo dai valori sul bordo\\ \\
%
Teo. Weierstrass \ \ \ DIM\\
Se $f:A\subseteq\mathbb{C}\rightarrow\mathbb{C}$ è olomorfa in A, allora è f analitica in A\\
Quindi $\forall z \in A \ e \ B_r(z_0)\subseteq A$ vale lo sviluppo \ $f(z)=\sum_{n=0}^{+\infty}c_n(z-z_0)^n \ \ \ c_n=\frac{1}{2\pi i} \int_{C_r(z_0)} \frac{f(z)}{(z-z_0)^{n+1}}dz$\\ \\
%
Oss. Inoltre per lo sviluppo di Taylor, sappiamo che $c_k=\frac{f^{(k)}(z_0)}{k!}$, da cui ricaviamo\\ la formula di Cauchy della derivata: \ \ $f^{(k)}(z_0)=\frac{k!}{2\pi i}\int_{C_r(z_0)}\frac{f(z)}{(z-z_0)^{k+1}}dz \ \ \forall k\in\mathbb{N}$ \\
%
%Lez5
%


\subsection{Singolarità delle funzioni di variabile complessa e Sviluppi di Lourent}
%
\begin{defi} 
Sia f olomorfa in A-$\{z_0\}$ diremo che $z_0$ è una \textbf{singolarità isolata}
\end{defi}
%
\begin{defi} Per analizzare le singolarità useremo le \textbf{serie di potenze bilatere}, dette anche sviluppi di Lourent\\
Ovvero $\sum_{-\infty}^{+\infty}a_n(z-z_0)^n = \sum_{n=1}^{+\infty}a_{-n}[(z-z_0)^{-1}]^n + \sum_{n=0}^{+\infty}a_n(z-z_0)^n$
\end{defi}
Oss. Il primo termine ha raggio di convergenza R' e converge in $|z-z_0|>\rho=\frac{1}{R'}$\\ \\
%
Oss. Se $0\le\rho <R\le+\infty$ la serie bilatera converge assolutamente in $A_{\rho,R}(z_0)=\{z\in\mathbb{C}:\rho<|z-z_0|<R\}$\\
e converge uniformemente in ogni E t.c. $\overline{E}\subset A_{\rho,R}(z_0)$\\ %ce ne vorrebbero due
%
Oss. Se $R=+\infty \ e \ a_n=0\ \forall n>0$ allora diremo che f è olomorfa all'infinito e $f(\infty)=a_0$\\ \\
%
%
Teo. Sia f olomorfa in una corona circolare $A_{\rho,R}(z_0)$, allora f è somma di una serie bilatera di potenze:\\
$f(z)=\sum_{-\infty}^{+\infty}a_n(z-z_0)^n$ \ \ dove $a_n=\frac{1}{2\pi i}\int_{\gamma}\frac{f(w)}{w-z_0}dw$ \ \ con $\gamma$ una circonferenza contenuta in $A_{\rho,R}(z_0)$\\
%
\begin{defi}Sia $f:A-\{z_0\}\subset\mathbb{C}\rightarrow\mathbb{C}$ olomorfa, allora $z_0$ è una singolarità:\\
\ • \ eliminabile se f è la restrizione di una funzione olomorfa in A\\
\ • \ polare se $lim_{z\rightarrow z_0}f(z)=+\infty$\\
\ • \ essenziale, altrimenti
\end{defi}
%
Prop. Sia $f:A-\{z_0\}\subset\mathbb{C}\rightarrow\mathbb{C}$ olomorfa, allora $\exists R>0$ t.c. $B_R(z_0)\subset A$, valgono:\\
\ • \ $a_{-m}=0 \ \forall m\ge 1 \implies z_0$ eliminabile\\
\ • \ $\exists$ un numero finito (non nullo) di $a_{-m}\neq 0 \implies z_0$ è un polo\\
\ • \ $\exists$ infiniti  $a_{-m}\neq 0 \implies z_0$ è essenziale\\ \\
%
Prop. Se $\exists m_0\ge 1 \ (m_0 \in\mathbb{N}_0)$ t.c.\ $a_{-m_0}\neq 0 \ e \ \ a_{-m}=0\ \forall m> m_0$ allora $z_0$ è un polo di ordine $m_0$\\
\phantom{Prop. }Se $\exists n_0\ge 1$ t.c. $a_{n_0}\neq 0 \ e \ \ a_n=0 \ \forall n<n_0$ allora $z_0$ è uno zero di ordine $n_0$\\ \\
%
Oss. Regola di De L'Hopital: se $z_0$ è un polo  o uno zero di f e g (olomorfe dove serve),\\ allora esistono finiti ed uguali \ \ $lim_{z\rightarrow z_0}\frac{f(z)}{g(z)}=lim_{z\rightarrow z_0}\frac{f'(z)}{g'(z)}$ \\


\subsection{Residui integrali}
%
Oss. Abbiamo detto che presa f olomorfa in $B_R(z_0)$ ($z_0$ al più escluso), allora $f(z)=\sum_{-\infty}^{+\infty}a_n(z-z_0)^n $\\
%
Teo. Sia $\gamma$ circonferenza interna a $B_R(z_0)$ ($\gamma$ contiene $z_0$), allora $\frac{1}{2 \pi i}\int_{\gamma}f(z)dz=a_{-1}$\\
%
\begin{defi} 
$a_{-1}$ è detto \textbf{residuo integrale} di f in $z_0$ e si indica con $Res(f,z_0)$
\end{defi}
%
%Lez6
%
Teo. (dei residui 1) \ Sia f olomorfa tranne al più in un numero finito di singolarità isolate $z_1,...z_N \in A$ \\
Se $\gamma$ è un circuito semplice in A contenente al suo interno tutte le singolarità, allora \ $\int_{\gamma}f(z)dz=2\pi i \sum_{j=1}^N Res(f,z_j)$\\ \\
%
Teo. (dei residui 2) \ Siano A un aperto esterno ad un circuito semplice $\gamma$ e f olomorfa in A  tranne al più in un numero finito di punti singolari $z_j$ e continua in $\overline{A}$, allora \ $\int_{-\gamma}f(z)dz=2\pi i [\sum_{j=1}^N Res(f,z_j) + Res(f,\infty)]$\\ \\
%
Cor. Sia $f:\mathbb{C}\rightarrow\mathbb{C}$ olomorfa in $\mathbb{C}$ tranne al più in un numero finito di singolarità,\\ allora la somma dei resuidi (compreso $\infty$) è zero\\ \\
%
%
Teo. (Principio di identità delle funzioni olomorfe)\\
Siano $f:A\subseteq\mathbb{C}\rightarrow\mathbb{C}$ olomorfa in A aperto connesso e $Z(f)$ l'insieme degli zeri di f, \\ allora f è identicamente nulla se e solo se $Z(f)$ ha punti di accumulazione in A\\ \\
%
Teo. (Unicità del prolungamento analitico)\\
Siano $f_0:S\subset A\subseteq\mathbb{C}\rightarrow\mathbb{C}$, A aperto connesso e S con un punto di accumulazione in A\\ se $\exists f:A\rightarrow\mathbb{C}$ olomorfa t.c. $f|_S=f_0$ allora f è unica\\
%
Oss. $f(z)=sin(z)$ ha zeri, ma il punto di accumulazione è $\infty$, che è esterno a $\mathbb{R}$\\ \\
%
Prop. I poli di una funzione f sono tutti e soli gli zeri di un prolungamento olomorfo di $\frac{1}{f}$\\ \\
%
Prop. Se f è olomorfa in un aperto connesso e non è identicamente nulla,\\ allora i suoi zeri sono tutti isolati e hanno ordine intero e finito\\ \\
%
%
Lemma di Jordan \ \ \ DIM\\
Siano $R,w>0$, \ $\rho:I\rightarrow(0,+\infty) \in \widetilde{C}^1$ con $I=[0,\pi]$ \ e \ $r_R=R\rho(t)e^{it} \ \ t\in I$ \\
Sia $C_R=(\gamma_R,r_R)$ un cammino con \ $\gamma_R=Im(r_R)$ \ e sia f continua su $\gamma_R$, allora:\\
$|\int_{C_R}e^{iwz} f(z)dz|\le \frac{c^*}{w}sup_{z\in\gamma_R}|f(z)|$ \ \ con $c^*$ indipendente da R\\ \\
%
Oss. Il lemma di Jordan vale anche nei seguenti casi:\\
\ • $\int_{C_R}e^{-iwz}f(z)dz \ I=[-\pi,0]$\ \ \
\ • $\int_{C_R}e^{wz}f(z)dz \ I=[\frac{\pi}{2},\frac{3}{2}\pi]$\ \ \
\ • $\int_{C_R}e^{-wz}f(z)dz \ I=[-\frac{\pi}{2},\frac{\pi}{2}]$\\
%
%Lez7
%
\subsection{Logaritmo e potenze di numeri complessi}
%
Oss. Dato $z\in\mathbb{C}_0$, esistono infiniti $w\in\mathbb{C} \ t.c. \ e^w=z$. Precisamente $w=ln|z|+i\theta \ con\ \theta\in Arg(z)$\\
dove $Arg(z):=\{\theta\in\mathbb{R} : z=|z|e^{i\theta} \}$ \ \ n.b. se $z=0,\  arg(z)=\mathbb{R}$\\
%
\begin{defi} 
La funzione \textbf{logaritmo principale} è $Ln(z)=ln|z| + iArg(z) \ con -\pi<Arg(z)<\pi$
\end{defi}
%
Oss. $z\rightarrow Ln(z)$ è definta su $E=\mathbb{C} -\{z\in\mathbb{C} :\ \mathcal{R}e(z)\le 0 ,\ \mathcal{I}m(z)=0 \} $\\
%
Oss. Su E $Ln(z)$ è olomorfa e vale $\frac{d}{dz} Ln(z)=\frac1z$, ma non posso estenderla a $\mathbb{C}$ perché c'è un salto\\
%
\begin{defi} 
$Ln(z)$ è una branca massimale di $ln(z)$, le altre sono $f_(z)=Ln(z) +i2k\pi  \ k\in\mathbb{Z}$
\end{defi}
%
Oss. Incollando i tagli delle branche generiamo una funzione continua, definita sulla superficie di Riemann\\
%
\begin{defi}Le funzioni simili a $ln(z)$, cioè insiemi di branche sono dette polidrome
\end{defi}
%
\begin{defi} 
I punti come z=0 per $ln(z)$, cioè punti intorno a cui girano tutte le branche, sono di diramazione
\end{defi}
\phantom{}\\
%
%
Potenze a esponente complesso \ \  $\alpha\in\mathbb{C} \ \ \ z^{\alpha}:=e^{\alpha lnz} \ \ \forall z\in\mathbb{C}_0$, \ \ analizziamo i casi: \vspace{0.08in}\\
%
I) $\alpha\in\mathbb{Z} \Longleftrightarrow \ e^{\alpha(lnz+2k\pi i)}=e^{\alpha lnz}$, in tal caso $z^{\alpha}$ è olomorfa in $\mathbb{C}_0$\vspace{0.08in}\\
%
II) $\alpha\in\mathbb{C}-\mathbb{Q} \ \ \implies z^{\alpha}:=\{w=|z|^{\mathcal{R}e(\alpha)} e^{i\mathcal{I}m(\alpha)ln|z|} e^{i\alpha\theta}, \theta\in Arg(z) \}$\\
Oss. In questo caso $z^{\alpha}$ ha infinite branche massimali olomorfe in \ $\mathbb{C}$\ -\ 
semiretta contenente l'origine\\
Oss. $z^{\alpha}$ è una funzione polidroma con punto di diramazione in z=0\\ 
Oss. Dall'equazione di Eulero $e^{i\pi}+1=0$ \ ricaviamo \ $\frac{ln(i)}{i}=\pi$
\vspace{0.08in}\\
%
III) $\alpha\in\mathbb{Q}-\mathbb{Z}$ \ \ \ in questo caso otteniamo un numero finito di branche massimali \\ 

\newpage



\section{Elementi di Analisi Funzionale}   


Sia $X\neq0$ uno spazio vettoriale su $\mathbb{R}$ (o su $\mathbb{C}$)
\begin{defi}
Una \textbf{norma} in X, che si indica con $||\cdot ||$, è una funzione $N:X\rightarrow [0,+\infty)$ t.c.\\
i) $N(x)\ge 0$  \ e $x=0\Leftrightarrow N(x)=0$ \ \ \ ii) $N(\alpha x)=\alpha N(x) \ \forall \alpha \in \mathbb{R}$\ \ \ iii) $N(x+y)\le N(x)+N(y) \ \forall x,y \in X$
\end{defi}

\begin{defi}
Uno S.V. X munito di norma $||\cdot||$ si dice \textbf{spazio vettoriale normato} e si indica con $(X,||\cdot||)$
\end{defi}
%
Oss. In $(X,||\cdot||)$ è sempre possibile definire una distanza ponendo $d(x,y):=||x-y||$\\
Dunque uno S.V.N. è sempre uno spazio metrico e quindi possiamo introdurre le nozioni topologiche viste\\


\subsection{Successioni negli spazi normati}

\begin{defi}
$\{x_n\}_{n\in\mathbb{N}}\subset X$ \textbf{converge} a $x\in X$ se \ \ $\forall\varepsilon>0,\ \exists n_0\in\mathbb{N} \ t.c. \ \ ||x_n-x||<\varepsilon \ \ \forall n>n_0$
\end{defi}

\begin{defi}
$\{x_n\}_{n\in\mathbb{N}}$ è \textbf{di Cauchy} se \ \ $\forall \varepsilon>0, \ \exists n_0\in\mathbb{N} \ t.c. \ \ ||x_n-x_m||<\varepsilon \ \forall n,m >n_0$
\end{defi}

\begin{defi}
$(X,||\cdot||)$ si dice \textbf{spazio di Banach} se rispetto alla distanza è completo,\\ cioè ogni successione di Cauchy è convergente
\end{defi}
\vspace{0.27cm}
%
%lez8
%
\subsection{Integrale di Lebesgue}
\begin{defi}
$R\subset\mathbb{R}^n$ è un plurirettangolo se $R=(a_1,b_1)\times...\times(a_n,b_n) \ \ a_j,b_j\in\mathbb{R} : a_j<b_j$ \\
La \textbf{misura} (o volume) n-dim di R è: $|R|_n=\prod_{j=1}^n(b_j-a_j)$
\end{defi}
\begin{defi}
$E\subset\mathbb{E}^n$ è di \textbf{misura nulla}: $|E|_n=0$ \ se $\forall\varepsilon>0, \ \exists\{R_j\}_{j\in\mathbb{N}}$ t.c.\\
i) $E\subseteq \bigcup_{j\in\mathbb{N}}R_j$ \ \ \ ii) $\sum_{j\in\mathbb{N}}|R_j|_n<\varepsilon$
\end{defi}
%
\phantom{}\\
%
Prop. Se $E\subset\mathbb{R}^n$ è t.c. $|E|_n=0$ allora $\forall F\subseteq E$ è t.c. $|F|_n=0$\\
%
Prop. $\{E_j\}_{j\in\mathbb{N}}$ t.c. $|E_j|_n=0 \ \forall j\in\mathbb{N} \ \implies \ |\bigcup_{j\in\mathbb{N}}E_j|_n=0$\\
%
Cor. $\forall E \subset \mathbb{R}^n$ numerabile \ $|E|_n=0$ \ \ \ \ \ (i.e. $|\mathbb{Q}^n|_n=0$)\\
%
\begin{defi}
Una proprietà $p(x) \ x \in \mathbb{R}^n$ vale \textbf{quasi ovunque} (q.o.) se\\ $\exists E \subset\mathbb{R}^n$ t.c. $|E|_n=0$ e p(x) è vera $\forall x\in\mathbb{R}^n/E$ 
\end{defi}

\begin{defi}
Una f a valori in $\mathbb{R}$ è definita q.o. in $\mathbb{R}^n$ se è definita su un insieme del tipo $\mathbb{R}^n/E$ con $|E|_n=0$
\end{defi}


\begin{defi}
Siano $R_1,...,R_k$ k plurirettangoli mutuamente disgiunti.\\
Una funzione \textbf{semplice} è una funzione del tipo $h(x)=\sum_{j=1}^k h_j\Chi_{R_j}$ \ con $h_j\in\mathbb{R}$
\end{defi}
Oss. É una combinazione lineare di funzioni caratteristiche, dove \ $\Chi_E=\begin{cases} 1 \ \text{ in E} \\ 0 \ \text{ altrove}\end{cases}$
\begin{defi}
Funzione $u:\mathbb{R}^n\rightarrow\mathbb{R}$ è \textbf{misurabile} se \ $\exists$ succesione di funzioni semplici t.c. $h_m\rightarrow u$ q.o. in $\mathbb{R}^n$
\end{defi}
%
Prop. Siano $u:\mathbb{R}^n\rightarrow I\subseteq\mathbb{R}$ \ misurabile e $f:I\rightarrow\mathbb{R}$ continua, \ allora $f\circ g$ è misurabile\\ \\
%
Oss. Perciò u,v misurabili $\implies u\cdot v, \ u\pm v, \ max\{u,v\}, \ min\{u,v\}$ misurabili\\
%
Prop. Sia $\{u_m\}_{m\in\mathbb{N}}$ successione di funzioni misurabili t.c. $u_m\rightarrow u$ q.o. allora u è misurabile
\begin{defi}
Se $h=\sum_{j=1}^n h_j\Chi_{R_j}$ è una funzione semplice, allora definiamo l'\textbf{integrale} \ $\int_{\mathbb{R}^n}h(x)dx:=\sum_{j=1}^nh_j|R_j|_n$
\end{defi}
\begin{defi}
Sia $u:\mathbb{R}^n\rightarrow\mathbb{R}$ misurabile, \ u è \textbf{intergabile secondo Lebesgue} in  $\mathbb{R}^n$ se\\ $\exists\{h_k\}_{k\in\mathbb{N}}$ successione di funzioni semplici t.c.\ \ \ \
i) $h_k\rightarrow u$ q.o. \ \ \ ii) $\{\int_{\mathbb{R}^n}h_k\}_{k\in\mathbb{N}}$ è di Cauchy
\end{defi}
%
Oss. ii) $\implies \exists \alpha\in\mathbb{R} \ t.c. \ lim_{k\to +\infty}\int_{\mathbb{R}^n}h_k=\alpha$\\
Si può provare che $\alpha$ non dipende dalla successione $\{h_k\}_{k\in\mathbb{N}}$, perciò possiamo porre $\int_{\mathbb{R}^n}u(x):=\alpha$\\ \\
%
%
Proprietà dell'integrale di Lebesgue:\\
1) u integrabile \ e \ v=u q.o. \ $\implies$ v integrabile e \ $\int_{\mathbb{R}^n}v =\int_{\mathbb{R}^n}u $\\
2) u integrabile \  $\implies$ \ |u| integrabile\\
3) u integrabile \ $\implies \ u^+, u^-$ integrabili\\
4) u,v integrabili \ $\implies$ \ max\{u,v\} e min\{u,v\} integrabili\\
5) u reale, non negativa e integrabile \ $\implies$ \ $\int_{\mathbb{R}^n}u\ge0$\\
6) u,v integrabili e  $u\le v$ q.o. \ $\implies \ \int_{\mathbb{R}^n}u \le \int_{\mathbb{R}^n}v$\\
7) u integrabile \ $\implies \ |\int_{\mathbb{R}^n}u|\le \int_{\mathbb{R}^n}|u|$\\
8) u,v integrabili \ $\implies \ \forall \alpha, \beta \in \mathbb{R} \ \ \alpha u +\beta v$ integrabile e vale $\int_{\mathbb{R}^n}\alpha u+\beta v = \alpha \int_{\mathbb{R}^n}u + \beta\int_{\mathbb{R}^n}v$\\ \\
%
%
Teo. Sia $u:\mathbb{R}^n\rightarrow\mathbb{R}$ misurabile, se $\exists\varphi$ positiva e integrabile t.c. $|u|\le\varphi$ q.o., allora u è integrabile\\ \\
%
%
Teo. Convergenza dominata per Lebesgue\\
Siano $u_k:\mathbb{R}^n\rightarrow\mathbb{R}$ integrabili t.c. \ \ i) $u_k\rightarrow u$ q.o. \ \ ii) $\exists\varphi:\mathbb{R}^n\rightarrow[0,\infty)$ integrabile t.c. $|u_k|\le\varphi$ q.o. \\ Allora u è integrabile e vale $\int_{\mathbb{R}^n}|u_k -u|\rightarrow 0$ per $k\to +\infty$ \ \ \ \ ovvero $lim_{k\to\infty}\int_{\mathbb{R}^n}u_k=\int_{\mathbb{R}^n}u$\\ \\
%
%
Teo. Convergenza monotona o Beppo Levi\\
Siano $u_k:\mathbb{R}^n\to\mathbb{R}$ integrabili e t.c. $u_{k+1}\ge u_k$ q.o. $\forall k \in \mathbb{N}$ \ \ allora: \\
i) $lim_{k\to+\infty}\int_{\mathbb{R}^n}u_k$ finito $\implies u_k\to u$ con u integrabile e $\lim_{k\to\infty}\int_{\mathbb{R}^n}|u_k-u|=0$\\
ii) $lim_{k\to+\infty}\int_{\mathbb{R}^n}u_k= +\infty \implies u_k\to u$ q.o. con u non integrabile, con $\int_{\mathbb{R}^n}u=+\infty$\\
iii) $\{u_k\}_{k\in\mathbb{N}}$ non converge q.o. a valori finiti\\


\subsection{Insiemi misurabili e integrali su insiemi misurabili}

\begin{defi}
$E\subset\mathbb{R}^n$ è \textbf{misurabile} (secondo Lebesgue) se $\Chi_E$ è misurabile \ \ \ $|E|_n=\begin{cases}+\infty \text{ se } \int_{\mathbb{R}^n}\Chi_E=+\infty \\ \int_{\mathbb{R}^n}\Chi_E \ \text{ se } \Chi_E \text{ è integrabile}\end{cases}$
\end{defi}
%
%
Proprietà della misura n-dim di Lebesgue $|\cdot|_n$:\\
• \  Numerabilmente additiva: \ $\forall \{E_j\}_{j\in\mathbb{N}}$ insieme misurabile e disgiunto, risulta $|\bigcup_{j\in\mathbb{N}}E_j|_n=\sum_{j\in\mathbb{N}}|E_j|_n$\\
• \ Invariante per traslazioni: \ $\forall E\subseteq\mathbb{R}^n$ misuabile, risulta \ $|E\pm a|_n=|E|_n \ \forall a\in\mathbb{R}^n$\\
%
\begin{defi}
Dati $E\subseteq\mathbb{R}^n$ misurabile e $u:E\to\mathbb{R}$\\
$u$ è L-integrabile in E se \ $\widetilde{u}(x):=u(x)\Chi_E$ è L-integrabile in $\mathbb{R}^n$, quindi sarà $\int_E u:=\int_{\mathbb{R}^n}\widetilde{u}$ 
\end{defi}
%
%
\phantom{}\\
%
Teo. Ogni funzione limitata in $\mathbb{R}^n$, nulla al di fuori di un compatto e R-integrabile\\
É L-integrabile e i due integrali coincidono\\
%
Oss. Esistono funzioni R-integrabili che non sono L-integrabili, come $f(x)=\frac{sinx}{x}$\\



\subsection{Spazi $\mathbf{L^P}$}


\begin{defi}
Sia $\Omega\subseteq\mathbb{R}^n$ L-misurabile, L'\textbf{insieme delle funzioni misurabili} in $\Omega$ è\\ $\mathcal{M}(\Omega):=\{u:\Omega\to\mathbb{R} \text{ misurabile}\}$
\end{defi}
%
%
Oss. Introduciamo la relazione di equivalenza: \ \ $u\sim v \Leftrightarrow u=v \ q.o. \ \ \forall u,v\in\mathcal{M}(\Omega)$\\
Oss. La classe di equivalenza con rappresentante u è \ \ $[u]:=\{v\in\mathcal{M}(\Omega)\ | \ v=u \ \ q.o. \ in \ \Omega\}$\\ \\
%
Oss. Consideriamo l'insieme quoziente: \ \ $M(\Omega)=\frac{\mathcal{M}(\Omega)}{\sim}=\{[u] : u\in\mathcal{M}(\Omega)\}$\\ \\
%
Oss. Se $\Omega$ aperto, è unica la funzione continua in ogni classe di equivalenza\\ 

\begin{defi}
Sia $\Omega\subseteq\mathbb{R}^n$ misurabile con $|\Omega|_n>0$ e $p\in[1,\infty)$ fissato, allora \\ 
$\mathbf{L^P(\Omega)}:=\{u\in M(\Omega): \ |u|^p \text{ è L-integrabile in }\Omega\}$
\end{defi}
%
Oss. $L^P(\Omega)$ possono essere dotati di una struttura di S.V.
Una buona norma è \ $||u||_p:=(\int_{\Omega}|u|^p)^{\frac{1}{p}}$\\ \\
%
Oss. Per verificare l'adeguatezza di questa norma, per $p\in(1,\infty)$ usiamo la seguente:\\
• \ Disuguaglianza Minkowski \ \ $(\int_{\Omega}|u+v|^p)^{\frac{1}{p}}\le(\int_{\Omega}|u|^p)^{\frac{1}{p}} + (\int_{\Omega}|v|^p)^{\frac{1}{p}}$ \ \ che si dimostra con la seguente:\\
• \ \ Disuguaglianza di H$\ddot{\text{o}}$lder \ \ se $|u|^p \ e \ |v|^q$ sono integrabili in $\Omega$ e \ $\frac1p +\frac1q=1$ allora $\int_{\Omega}|uv|\le(\int_{\Omega}|u|^p)^{\frac{1}{p}}(\int_{\Omega}|v|^q)^{\frac{1}{q}}$\\ \\
%
%
Teo. $(L^P(\Omega), ||\cdot||_p)$ è uno spazio di Banach \ \ $\forall p\in[1,\infty)$\\

\begin{defi}
$\mathbf{L^{\infty}(\Omega)}:=\{u\in M(\Omega) \ | \ \ \exists K\ge 0 \ t.c. \ |u|\le K \ q.o. \ in \ \Omega \}$
\end{defi}
%
Oss. $L^{\infty}(\Omega)$ è uno S.V. su $\mathbb{R}$\\ \\
%
%
Prop. $\forall u \in L^{\infty}(\Omega) $ esiste l'estremo superiore essenziale $\alpha=min\{K\ge0 \ t.c. \ |u|\le K \ q.o. \ in \ \Omega\}$\\
Oss. Una norma in $L^{\infty}(\Omega)$ $||u||_{\infty}:= ess\ sup_{\ \Omega}\ |u(x)|$\\
Teo. $(L^{\infty}(\Omega),||\cdot||_{\infty})$ è uno spazio di Banach\\ \\
%
%
Oss. La disuguaglianza di H$\ddot{\text{o}}$lder si estende facilmente a $p=\infty$ \ \ $\int_{\Omega}|uv|\le\int_{\Omega}(ess\ sup \ |u|)|v|\le ||u||_{\infty}\cdot||v||_1$\\ \\
%
%
Oss. A questo punto abbiamo una famiglia infinita di spazi di Banach infito dimensionali, dotati di una norma, ora introduciamo nozione di prodotto scalare e quindi di spazio hilbertiano\\ \\






\subsection{Spazi di Hilbert}

\begin{defi}
Sia X uno S.V. su $\mathbb{R}$ \ \  $p:X\times X\to \mathbb{R}$ si dice \textbf{prodotto scalare} in X se:\\
i) $p(x,x)\ge 0$ e $p(x,x)=0 \Leftrightarrow x=0$\\
ii) $p(x,y)=\overline{p(y,x)} \ \ \forall x,y \in X$\\
iii) $p(\alpha x +\beta y, u)=\alpha p(x,u) + \beta p(y,u) \ \ \forall \alpha, \beta \in \mathbb{R} \ \forall x,y,u \in X$
\end{defi}
%
Oss. $p(x,y)= \ <x,y>$ \ \ \ \ $(X,\ <\cdot,\cdot>
)$ \ si dice spazio \textbf{pre-Hilbertiano}\\ \\
%
Oss. Disuguaglianza di Cauchy-Schwarz $|<x,y>|\le \sqrt{<x,x>}+\sqrt{<y,y>} \ \ \forall x,y \in X$\\ \\
%
%
Prop. $||x||:=\sqrt{<x,x>}$ è la norma indotta dal prodotto scalare\\

\begin{defi}
$(X,\ <\cdot,\cdot>)$ è uno \textbf{spazio di Hilbert} se è completo rispetto alla norma indotta
\end{defi}





\subsection{Ortogonalità}
%
Oss. C.-S. $\implies -1\le\frac{<x,y>}{||x||\cdot||y||}\le 1 \ \implies \exists !\ \theta\in [0,\pi] \ t.c. \ cos\theta=\frac{<x,y>}{||x||\cdot||y||} \ \ \ \theta$ è l'angolo tra i vettori x e y

\begin{defi}
x è \textbf{ortogonale} a y se \ $<x,y>\ =0$
\end{defi}
%
Teo. (Neumann): \ Se $(X,||\cdot||)$ è di Banach, allora $||\cdot||$ è indotto da un prodotto scalare sse vale\\ l'identità del parallelogramma: \ $||x-y||^2+||x+y||^2=2||x||^2+2||y||^2 \ \ \forall x,y\in X$ \\


\subsection{Sistemi e basi ortonormali}


\begin{defi}
Dato uno spazio pre-Hilbertiano $(H, \ <\cdot,\cdot>)$ su $\mathbb{R}$ \\ $\{u_n\}_{n\in E} \ \ E\subseteq \mathbb{N}$ è un \textbf{sistema ortonormale} di H se\ \  $<u_r,u_s>=\begin{cases}1 \ \ r=s \\ 0 \ \ r\ne s\end{cases}$
\end{defi}
%
Oss. Ovvero se è ortogonale e tutti i suoi elementi hanno norma unitaria: \ $||u_j||=1 \ \ \forall j\in E$\\ \\
%
Prop. Se $\{u_j\}_{j\in E}$ è un sistema ortonormale in H Hilbert, allora:\\
• \ Se E non è finito \ \ $\sum_{j\in E}c_ju_j$ converge in H $\Leftrightarrow \sum_{j\in E}|c_j|^2<\infty \ \ \ \forall\{c_j\}_{j\in E}\subset\mathbb{R}$\\
• \ $u=\sum_{j\in E}c_ju_j \in H \ \ \ c_j=\ <u,u_j>$\\
• \ Identità di Parseval: \ $x=\sum_{j\in E}c_jx_j \ \ y=\sum_{j\in E}d_jy_j \ \implies <x,y>\ =\sum_{j\in E}c_jd_j$\\


\begin{defi}
Una \textbf{base ortonormale} di $H$ Hilbert è un sistema ortonormale $\{u_j\}_{j\in E} \ t.c. \\ \forall x\in H \ \ \exists ! \ \{c_j\}_{j\in E}\subset\mathbb{R} \ t.c. \ x=\sum_{j\in E} c_ju_j$
\end{defi}
%
Oss. Se E infinito, allora esiste sempre una base ortonormale\\

\begin{defi}
Dato $(H, \ <\cdot,\cdot>)$ Hilbert e $\{u_j\}_{j\in E}$ un suo sistema ortonormale con E infinito, allora ogni $x\in H$ è somma della \textbf{serie di Fourier} astratta $\sum_{j\in E}x_ju_j$ con $x_j= \ <x,u_j> \ =$ coefficienti di Fourier
\end{defi}
%
Oss. Se E è finito, non è più una "serie", è una somma finita equivalente a una proiezione ortogonale\\


\subsection{Serie di Fourier in $\mathbf{L^2}$}

\begin{defi}
Dati $H=L^2([-\pi,\pi])$ in $\mathbb{R}$ e $<f,g>\ :=\int_{-\pi}^{\pi}fg$ \ \ $(H,\ <\cdot,\cdot>)$ è di Hilbert \\ $\{\frac{1}{\sqrt{2\pi}},\frac{cos(nt)}{\sqrt{\pi}},\frac{sin(nt)}{\sqrt{\pi}}\}_{n\in\mathbb{N}_0}$ è una base ortornomale di H
\end{defi}
%
Oss. Per scrivere i coefficienti di Fourier è necessario $f\in L^1([-\pi,\pi])$ \\ \\
%
Oss. Posti \ $a_0=\ <f,\frac{1}{\sqrt{2\pi}}>,\ \ a_n=\ <f,\frac{cos(nt)}{\sqrt{\pi}}>,\ \ b_n=\ <f, \frac{sin(nt)}{\sqrt{\pi}}>$\\
$T_N(t)= \frac{a_0}{\sqrt{2\pi}}+\sum_{n=1}^N a_n\frac{cos(nt)}{\sqrt{\pi}} +\sum_{n=1}^N b_n\frac{sen(nt)}{\sqrt{\pi}}$ è detto polinomio trigonometrico\\
$\{T_N(t)\}_{N\in\mathbb{N}}$ è detta serie trigonometrica, se converge a una f, allora si dirà serie di  Fourier di f\\ \\
%
%
Prop. Se $\sum_{n=1}^{+\infty}|a_n|, \ \sum_{n=1}^{+\infty}|b_n|$ convergono, allora la serie di Fourier di f  converge ass. e unif. in $[-\pi,\pi]$\\ 

\begin{defi}
$f:[a,b]\subset\mathbb{R}\to\mathbb{R}$ è \textbf{continua a tratti} in $[a,b]$ se è continua in $[a,b]$ tranne al più in un numero finito di punti, dove esistono finiti il limite destro e sinistro
\end{defi}

\begin{defi}
$f:[a,b]\subset\mathbb{R}\to\mathbb{R}$ soddisfa la \textbf{condizione di Dirichlet} $\mathbb{D}$ in $x_o\in(a,b)$ se vale una delle seguenti:\\
i) f è derivabile in $x_0$  \ \ \ \ ii) f è continua in $x_0$ e ha un punto angoloso in $x_0$ \\ iii) f ha un salto in $x_0$ ed $\exists$ finiti $lim_{x\to x_0^-} \frac{f(x)-f(x_0^-)}{x-x_0}$ e $lim_{x\to x_0^+} \frac{f(x)-f(x_0^+)}{x-x_0}$
\end{defi}
%
%
Teo. (Convergenza puntuale)\\
Sia $f:[-\pi,\pi]\subset\mathbb{R}\to\mathbb{R}$ continua a tratti, allora la sua serie di Fourier converge in ogni $x_0\in(-\pi,\pi)$ in cui è soddisfatta la condizione $\mathbb{D}$, più precisamente converge a $\frac{f(x_0^-)+f(x_0^+)}2$\\ \\
%
%
Teo. (Convergenza uniforme)\\
Sia $f:[a,b]\subset\mathbb{R}\to\mathbb{R}$ continua in $[-\pi,\pi]$ con derivata continua, tranne al più un numero finito di punti nei quali vale la condizione $\mathbb{D}$, allora la serie F di f converge a f assolutamente e uniformemente in $[-\pi,\pi]$\\
Oss. In particolare $f\in C^1([-\pi,\pi]) \implies$ serie F converge ass e unif in $[-\pi,\pi]$\\ \\
%
%
Teo. (Carleson)\\
Se $f\in L^2([-\pi,\pi])$ allora la sua serie di Fourier converge puntualmente quasi ovunque\\
%
Oss. In un insieme finito \ f continua a tratti $\implies f \in L^2$\\


\begin{defi}
Dati $H=L^2([-\pi,\pi])$ in $\mathbb{C}$ e \ $<f,g>\ :=\int_{-\pi}^{\pi}f\overline{g}$ \ \ $(H,\ <\cdot,\cdot>)$ è di Hilbert \\ $\{\frac{e^{int}}{\sqrt{2\pi}}\}_{n\in\mathbb{Z}}$ è una base ortornomale di H
\end{defi}
%
Oss. $f\in L^1([-\pi,\pi]) \implies \gamma_n=\frac{1}{\sqrt{2\pi}}\int_{-\pi}^{\pi}f(x)e^{-inx}dx \ \ n\in\mathbb{Z} \ \implies$ la serie di Fourier è $\sum_{n\in\mathbb{Z}}\gamma_n e^{inx}$\\ \\
%
Prop. $f\in L^p([-\pi,\pi]), \ p\in (1,\infty] \implies$ serie di F di f converge q.o.\\
Oss. Ovvero la successione $\sum_{n=-N}^N \gamma_n \frac{e^{inx}}{\sqrt{2\pi}}$ converge per $N\to +\infty$ a f in $L^p$\\ \\
%
%
Oss. f derivabile con al più un num finito di punti angolosi $\implies$ serie F converge uniformemente a f\\

\begin{defi}
Un insieme A è denso in B se $\forall x \in B \ \ \exists \{u_n\}_{n\in\mathbb{N}}\subset A \ \ t.c. \ u_n\to x$ in B
\end{defi}
Oss. $C^{\infty}(\Omega)$ è denso in $L^P(\Omega) \ \forall p\in [1,\infty) $
\newpage




\section{Elementi di Teoria delle Distribuzioni}

\begin{defi}
Sia $v:\Omega\to\mathbb{R},\ \Omega\subseteq\mathbb{R}^n$ aperto, non nullo e v continua\\
Il \textbf{supporto} di v è l'insieme chiuso in $\mathbb{R}^n$ \ $supp(v):=\overline{\{x\in\Omega : \ v(x)\neq 0\}}$
\end{defi}



\begin{defi}
Sia $\Omega\subset\mathbb{R}^n$ aperto di $\mathbb{R}^n$, allora $\mathbf{C^{\infty}(\Omega)}$ è lo S.V. su $\mathbb{R}$ delle funzioni $v:\Omega\to\mathbb{R}$ t.c. le derivate $D^{\alpha}v$ sono continue in $\Omega \ \forall \alpha\in\mathbb{N}^n$ multi-indice 
\end{defi}
%
Oss. $\alpha=(\alpha_1,...,\alpha_n)\in\mathbb{N}^n$ e $D^{\alpha}v(\underline{x})=D^{\alpha_1}_{x_1},...,D^{\alpha_n}_{x_n}$ di $v(\underline{x})$ \ \ \ $D^i_x$ è la derivata i-esima rispetto a x\\


\begin{defi}
$C^{\infty}_c(\Omega)=\{v\in C^{\infty}(\Omega) : \ supp(v) \text{ è compatto in } \mathbb{R}^n\}$ è uno S.V. su $\mathbb{R}$
\end{defi}

\begin{defi}
\textbf{Convergenza successionale}: \ $\{\varphi_j\}_{j\in\mathbb{N}}\subset C_c^{\infty}(\Omega)$ converge a $\varphi\in C_c^{\infty}(\Omega)$ se\\
$\exists K \subset\mathbb{R}^n$ compatto t.c. $supp(\varphi_j)\subseteq K \ \forall j \in \mathbb{N}$ \ \ e \ $\{D^{\alpha}\varphi_j\}_{j\in\mathbb{N}}$ \ converge unif. in $\Omega$ a $D^{\alpha}\varphi \ \forall\alpha\in\mathbb{N}^n$
\end{defi}
%
Prop. Se $\{\varphi_j\}_{j\in\mathbb{N}}\subset C_c^{\infty}(\Omega)$ soddisfa (i) e $\{D^{\alpha}\varphi_j\}_{j\in\mathbb{N}}$ è di Cauchy uniformemente $\forall\alpha\in\mathbb{N}^n$ \ allora \\
$\exists \phi\in C^{\infty}_c(\Omega)$ t.c. $\{\phi_j\}_{j\in\mathbb{N}}$ converge a $\phi$, \ cioè abbiamo completezza

\begin{defi}
$\mathcal{D}(\Omega)$ è lo S.V. $C^{\infty}_c(\Omega)$ munito della convergenza successionale
\end{defi}
\vspace{0.2cm}

\subsection{Spazio Duale di uno Spazio Vettoriale}

\begin{defi}
Sia X uno S.V. su $\mathbb{R}$, il \textbf{duale algebrico} di X è lo S.V. X':=$\{ L:X\to \mathbb{R} \ |\ \text{L funz. lineare} \}$ 
\end{defi}
%
Oss. $\phi_j\to\phi$ in $\mathcal{D}(\Omega)$ significa che vale la convergenza successionale 

\begin{defi}
$u:\mathcal{D}(\Omega)\to\mathbb{R}$ è una \textbf{distribuzione} su $\Omega$ o funzione generalizzata se valgono: \ \ i) è lineare\\
ii) $\forall \{\phi_j\}_{j\in\mathbb{N}}\subset\mathcal{D}(\Omega)$ t.c. $\phi_j\to\phi$ in $\mathcal{D}(\Omega)$ \ \ \ si ha $u(\phi_j)\to u(\phi)$ per $j\to\infty$
\end{defi}

\begin{defi}
Lo \textbf{spazio delle distribuzioni} su $\Omega$ è lo S.V. \ $\mathcal{D}'(\Omega):=\{u: C^{\infty}_c(\Omega)\to\mathbb{R} \ | \ \text{soddisfano i) e ii)}\}$
\end{defi}

\begin{defi}
$\{u_j\}_{j\in\mathbb{N}}\subset\mathcal{D}'(\Omega)$ converge a $u\in\mathcal{D}'(\Omega) \Longleftrightarrow \ u_j(\phi)\to u(\phi)$ per $j\to\infty \ \forall \phi \in \mathcal{D}(\Omega)$   
\end{defi}
%
Oss. $\mathcal{D}'(\Omega)$ è completo: \\
$\{u_j\}_{j\in\mathbb{N}}\subset\mathcal{D}'(\Omega)$ t.c. $\{u_j(\phi)\}_{j\in\mathbb{N}}$ è di Cauchy $\forall \phi\in\mathcal{D}(\Omega) \ \implies \ \exists u\in\mathcal{D}'(\Omega)$ t.c. $u_j\to u$ in $\mathcal{D}'(\Omega)$\\ \\ 
%
Notazione: \ • \ $<u,\phi>:=u(\phi)$ \ \ \ \ • \  $\mathcal{D}'(\Omega)$ si sottointenderà  dotato di convergenza puntuale   \\


\begin{defi}
$L^1_{loc}(\Omega)$:=\{$f\in \mathcal{M}(\Omega) : \ f|_K\in L^1(K) \ \ \forall K\subset\Omega, \ K \text{ compatto}$\} \ con $\Omega\subset\mathbb{R}^n$ aperto, non nullo
\end{defi}
%
Oss. Siano $f\in L^1_{loc}(\Omega)$ e $u_f\in\mathcal{D}'(\Omega)$ la distribuzione associata a f, allora \ $<u_f,\phi>:=\int_{\Omega}f\phi \ dx$\\ \\
%
%
Proprietà:\ \
\ • \ $\int_{\Omega} f\phi \ dx \le \ ||\phi||_{\infty} \ \int_{supp(\phi)}|f| \ dx \ \ \in \mathbb{R}$ \ \ \
\ • \ $u_f : \mathcal{D}(\Omega)\to\mathbb{R}$ ben definita e lineare\\
\hspace*{0.7in} • \ se $\phi_j\to\phi$ in $\mathcal{D}(\Omega)$ allora \  $<u_f,\phi_j> \ \to \ <u_f,\phi> $ \  in $supp(\phi_j)$\\ \\
%
%
Lemma di annullamento: \ \ Sia $f\in L^1_{loc}(\Omega) $ allora $ \int_{\Omega}f\phi \ dx = 0 \ \forall\phi\in\mathcal{D}(\Omega) \implies f=0$ q.o. in $\Omega$\\
Oss. L'applicazione $\mathcal{F}:L^1_{loc}(\Omega)\to \mathcal{D}'(\Omega) \ \ f \mapsto u_f$ \ \ è ben definitiva e iniettiva per il lemma\\
Perciò $L^1_{loc}(\Omega)$ può essere identificata con  $\mathcal{F}(L^1_{loc}(\Omega))\subset\mathcal{D}'(\Omega)$ ed è indifferente scrivere $f$ o $u_f$\\ \\
%
%
Oss. $\mathcal{F}(L^1_{loc}(\Omega))\ne\mathcal{D}'(\Omega)$ \ \ \ Esempi:\\ \\
%
• \ $f(t)=\frac1t$ per $t\neq 0 \implies f\not\in L^1_{loc}(\mathbb{R})$ \ \ \ però $u_f=v.p.\frac1t \ \in \mathcal{D}'(\Omega)$ \\
\phantom{• \ }$<u_f,\phi>:= v.p. \int_{\mathbb{R}}\frac{\phi(t)}{t}dt = lim_{\eps\to 0^+} \int_{|t|>\eps}\frac{\phi(t)}{t}dt \ =\int_{\mathbb{R}}\frac{\phi(t)-\phi(0)}{t} dt \ \ \forall \phi\in\mathcal{D}(\mathbb{R})$\\ \\
%
%
Teo. Se $u:\mathcal{D}\to\mathbb{R}$ è lineare, allora \ \  $u\in\mathcal{D}'(\Omega)\Leftrightarrow \forall K\subset\Omega$ \ K compatto, $\exists \ C_K>0$ e $m_k\in\mathcal{N}$ t.c. $|<u,\phi>|\le C_k \sum_{|\alpha|<m_K} ||D^{\alpha}\phi||_{\infty} \ \ \forall\phi\in\mathcal{D}(\Omega)$ t.c. $supp(\phi)\subseteq K$\\ \\
%
Oss. Se $m_k$ è indipendente da K, allora m si dirà ordine della distribuzione \\
se m=0 le distribuzioni si diranno misure\\


\subsection{Derivata di una distribuzione}

Oss. Sia $f\in L^1(\mathbb{R})\subset L^1_{loc}(\mathbb{R})$ t.c. $f'\in C^0(\mathbb{R})\subset L^1_{loc}(\mathbb{R})$ allora\\
$<u_{f'},\phi>\ =\int_{\mathbb{R}}f'\phi dt = - \int_{\mathbb{R}} f \phi' dt$ \ \ \ ma questo non richiede $f\in C^1(\mathbb{R})$

\begin{defi}
Se $u\in\mathcal{D}'(\mathbb{R})$ allora la \textbf{derivata distribuzionale} di u è la distribuzione \ $Du = v $ t.c.\\ $ \ <v,\phi> \ :=\ -<u,\phi'> \ \ \forall \phi\in \mathcal{D}(\mathbb{R})$
\end{defi}
%
Oss. Se $f\in C^1(\mathbb{R})$ avremo $u_{f'} = Du_f$\\ \\
%
%
Oss. In generale se $f\in C^{\infty}(\mathbb{R}^n)$ e $\alpha\in\mathbb{R}^n$ \ \ $\int_{\mathbb{R}^n}D^{\alpha} f \ \phi \ dx = (-1)^{|\alpha|}\int_{\mathbb{R}^n}fD^{\alpha}\phi \ dx$, quindi
\begin{defi}
La derivata ditribuzionale di ordince $\alpha $ di $u\in\mathcal{D}'(\mathbb{R}^n)$ è la distribuzione $D^{\alpha}u=v$ \ t.c.\\ $<v,\phi>:=(-1)^{\alpha}\int_{\mathbb{R}^n}fD^{\alpha}\phi \ dx$ 
\end{defi}
%
Oss. Se $f\in C^k(\mathbb{R}^n)$ allora $u_{D^{\alpha}f}=D^{\alpha}u_f \ \ \forall \alpha: |\alpha|\le k$ \\ \\
%
%
%Oss. Fissati u e $\alpha$, la derivata distribuzionale di $\phi$ di ordine $\alpha$ 
%
Proprità:\\
• \ $u_n\to u$ in $\mathcal{D}'(\Omega) \implies \ D^{\alpha}u_n\to D^{\alpha}u$ in $\mathcal{D}'(\Omega) \ \ \forall\alpha\in\mathbb{N}^n$\\
• \ Se \ $\sum_{n\in\mathbb{N}}u_n$ converge a u in $\mathcal{D}'(\Omega)$, cioè $<\sum_{n\in\mathbb{N}}u_n,\phi>\to<u,\phi> \ \forall \phi\in\mathcal{D}(\Omega)$\\
\phantom{• \ }allora \ $\sum_{n\in\mathbb{N}}D^{\alpha}u_n \to D^{\alpha}u$ in $\mathcal{D}'(\Omega) \ \ \forall\alpha\in\mathbb{N}^n$\\
• \ $<u(Ax+\underline{b}),\phi>:=<u,\phi(A^{-1}(y-\underline{b}))\cdot(detA)^{-1}>$ \ \ con $u\in\mathcal{D}'(\mathbb{R}^n), A\in\mathbb{R}^{n\times n}, detA\ne 0, \underline{b}\in\mathbb{R}^n$\\
\phantom{• \ }In particolare dato un versore $\underline{v}\in\mathbb{R}^n$ \ \ $<D^{\underline{v}}u,\phi>\ =-<u,D^{\underline{v}}\phi> \ \ \forall\phi\in\mathcal{D}(\mathbb{R}^n)$ \\ \\
%
Oss. Ogni $u\in\mathcal{D}'(\mathcal{R})$ ha una primitiva in $\mathcal{D}'(\mathcal{R})$, ovvero:\\
Teo. $\forall u\in\mathcal{D}'(\mathcal{R}), \ \exists v\in\mathcal{D}'(\mathcal{R})$ t.c. $Dv=u$, \ inoltre $\forall \Tilde{v}=v+c \ con \ c\in\mathbb{R} \implies D\Tilde{v}=u$\\ \\
%
%
Oss. $f\in L^1(\mathbb{R})$ t.c. $\exists g\in L^1(\mathbb{R})$ : $||\frac{f(\cdot +h_n)-f(\cdot)}{h_n}-g(\cdot)||_{L^1(\mathbb{R})}\to 0  \ \forall\{h_n\}\subset\mathbb{R} : h_n\to 0 \ \implies Df=g$\\ \\
%
Oss. $u\in\mathcal{D}'(\mathbb{R})$ è T-periodica con $T\ne 0$ se $<u,\phi(\cdot)>\ =\ <u,\phi(\cdot-T)> \ \forall\phi\in\mathcal{D}(\mathbb{R})$ \\ \\
%
%
Oss. Teorema di Schwarz\\
$\forall u\in\mathcal{D}'(\Omega), \ \Omega\subseteq\mathcal{R}^n$ aperto, non nullo \ \ \ $\frac{\partial^2 u}{\partial x_i\partial x_j}=\frac{\partial^2 u}{\partial x_j\partial x_i} \ \ \forall i,j \in\{1,...,n\}$\\



\subsection{Distribuzioni temperate}

\begin{defi}
Lo \textbf{spazio di Schwarz} o delle funzioni a decrescita rapida è:\\ $\mathcal{S}(\mathbb{R}^n):=\{\phi\in C^{\infty}(\mathbb{R}^n): D^{\alpha}\phi= o(|x|^{-k}) \ \text{ per }|x|\to\infty \ \ \forall \alpha \in \mathbb{N}^n, \forall k\in\mathbb{N}\}$
\end{defi}
%
\phantom{}\\
Oss. $\mathcal{S}(\mathbb{R}^n)$ è uno S.V. in cui possiamo introdurre una convergenza successionale, per cui $\mathcal{S}(\mathbb{R}^n)$ è completo \\ $\phi_n\to\phi$ in $\mathcal{S}(\mathbb{R}^n)$ per $n\to\infty \Longleftrightarrow \ |x|^kD^{\alpha}\phi_n\to |x|^kD^{\alpha}\phi$ uniformemente $\forall \alpha\in\mathbb{N}^n \ \forall k\in \mathbb{N}$\\

\begin{defi}
Sia $u:\mathcal{S}(\mathbb{R}^n)\to\mathbb{R}$ funzione lineare, \ u è \textbf{continua} se $<u,\phi_n>\ \to \ <u,\phi> \ \ \forall ...$
\end{defi}

\begin{defi}
Lo \textbf{spazio delle distribuzioni temperate} è \ $\mathcal{S}'(\mathbb{R}^n):=\{u:\mathcal{S}(\mathbb{R}^n)\to\mathbb{R} \ \text{lineare e continua}\}$
\end{defi}
%
Oss. $\mathcal{D}(\mathbb{R}^n)\subset\mathcal{S}(\mathbb{R}^n)$, ma $\mathcal{D}(\mathbb{R}^n)\ne\mathcal{S}(\mathbb{R}^n)$ \ \ \ un esempio è $e^{-||x||_2^2}$\\
Oss. $u\in\mathcal{S}'(\mathbb{R}^n) \ \implies u|_{\mathcal{D}(\mathbb{R}^n)}\in\mathcal{D}'(\mathbb{R}^n)$\\ \\
%
%
Teo. $u\in\mathcal{D}'(\mathbb{R}^n)$ si può estendere a $\Tilde{u}\in\mathcal{S}'(\mathbb{R}^n)$ se e solo se\\ $\forall\{\phi_n\}_{n\in\mathbb{N}}\subset\mathcal{D}(\mathbb{R}^n)$ t.c. $\phi_n\to 0$ in $\mathcal{S}(\mathbb{R}^n)$ si ha $<u,\phi_n>\to 0 = \ <u,0>$\\
Oss. Equivale alla continuità di u in 0 in $\mathcal{S}(\mathbb{R}^n)$, che è sufficiente per estendere u a $\Tilde{u}$ per il lemma di densità\\ \\
%
%
Lemma di densità: \ $\forall v\in\mathcal{S}(\mathbb{R}^n) \ \ \exists \{u_n\}_{n\in\mathbb{N}}\subset\mathcal{D}(\mathbb{R}^n)$ t.c. \ $u_n\to v$ in $\mathcal{S}(\mathbb{R}^n)$ per $n\to\infty$\\
%
%
%Oss. $u_{e^x}\not\in \mathcal{S}'(\mathbb{R})$\\


\subsection{Prodotto distribuzione-funzione}

Prop. $u\in\mathcal{D}'(\Omega),\ \psi\in C^{\infty}(\Omega) \ \ <u\psi,\phi>\ := \ <u,\psi\phi> \ \forall\phi\in\mathcal{D}(\Omega) \ \implies u\psi\in\mathcal{D}'(\Omega)$\\
%
Prop. $u\in\mathcal{S}'(\mathbb{R}^n),\ \psi\in \mathcal{S}(\mathbb{R}^n) \ \ <u\psi,\phi>\ := \ <u,\psi\phi> \ \forall\phi\in\mathcal{S}(\mathbb{R}^n) \ \implies u\psi\in\mathcal{S}'(\mathbb{R}^n)$\\ \\
%
%
Oss. Per n=1 vale la regola di Leibniz per il prodotto \ \ $D^{\alpha}(u\psi)=\sum_{k\le\alpha}\binom{\alpha}{k}\ D^ku\ D^{\alpha-k}\psi$\\ \\
%
%
\textbf{Problema di divisione}:\\
Data $v\in\mathcal{D}'(\Omega)$ e $\psi\in C^{\infty}(\Omega)$ trovare $u\in\mathcal{D}'(\Omega)$ t.c. $\psi u=v$\\

\newpage



\section{Trasformata di Fourier}

\begin{defi}
Sia $u\in L^1(\mathbb{R}^n)$, la \textbf{trasformata di Fourier} di u è la funzione \ $\hat{u}:\mathbb{R}^n\to\mathbb{C}$ \ \ $\hat{u}(\xi):=\int_{\mathbb{R}^n}e^{-i\xi x}u(x)\ dx$
\end{defi}
%
%
Oss. $|\hat{u}(\xi)|\le||u||_{L^1(\mathbb{R}^n)} \ \forall \xi\in\mathbb{R}^n$ \ \ \ quindi $\hat{u}$ è ben definita e limitata\\
%
Oss. $\mathcal{F}:L^1(\mathbb{R}^n)\to L^{\infty}(\mathbb{R}^n) \ \ u\mapsto \hat{u}$ \ è ben definita e lineare\\ \\
%
%
%
%
Proprietà: \ \ \ $\forall u\in L^1(\mathbb{R}^n), \ \forall a \in \mathbb{R}_0, \ \forall x_0\in\mathbb{R}^n$\\
%
1) $v(x)=u(ax) \implies \hat{v}(\xi)=\frac{1}{|a|}\hat{u}(\frac{\xi}{a})$\\
%
2) $v(x)=u(x-x_0) \implies \hat{v}(\xi)= e^{-ix_0\xi}\hat{u}(\xi)$\\
%
3) $v(x)=e^{ix_0x}u(x) \implies \hat{v}(\xi)= \hat{u}(\xi-x_0)$\\
%
4) $v(x)=u(A^{-1}x) \ A\in\mathbb{R}^{n\times n}, detA\ne0 \ \implies \hat{v}(\xi)=|detA|\ \hat{u}(A^T\xi)$\\
%
5) $v(x)=\overline{u(x)} \implies \hat{v}(\xi)=\overline{ \hat{u}(-\xi)}$\\
%
6) u pari (dispari) $\implies \hat{u}$ pari (dispari)\\
%
7) u reale pari $\implies \hat{u}$ reale pari \ \ \ \ u reale dispari $\implies \hat{u}$ immaginaria dispari\\
%
Oss. Per la 4) vale: \ u radiale $\implies \hat{u}$ radiale\\ \\
%
%
Lemma di Riemann-Lebesgue: \ \ \ DIM\\
$\forall u\in L^1(\mathbb{R}^n)$ risulta \ \ $\hat{u}\in C^0(\mathbb{R}^n)$ \ e \ \ $lim_{|\xi|\to \infty} \ \hat{u}(\xi)=0$\\



\subsection{La trasformata di Fourier e la derivazione}
%
Teo.1 Sia $u\in L^1(\mathbb{R})$ t.c. \ $xu\in L^1(\mathbb{R})$ allora $\hat{u}\in C^1(\mathbb{R})$ e $\frac{\partial}{\partial\xi}\hat{u}(\xi)=-i\widehat{(xu)}(\xi)$\\ \\
%
Teo.2 Sia $u\in C^1(\mathbb{R})$ t.c. \ $u, \ u' \in L^1(\mathbb{R})$ allora $\hat{u}'(\xi)=i\xi\hat{u}(\xi)$\\ \\
%
Teo.3 Sia $u\in L^1(\mathbb{R}^n)$ t.c. \ $x_ju\in L^1(\mathbb{R}^n)$ allora $\exists \partial_{\xi_j}\hat{u}\in C(\mathbb{R}^n)$ e $\partial_{\xi_j}\hat{u}(\xi)=-i\widehat{(x_ju)}(\xi)$\\ \\
%
Teo.4 Sia $u\in C^1(\mathbb{R}^n)$ t.c. \ $u, \ \partial_{x_j}u \in L^1(\mathbb{R}^n)$ allora $\widehat{(\partial_{x_j}u)}(\xi)=i\xi_j\hat{u}(\xi)$\\ \\
%
%
Oss. Teo.3 (1)\ $\implies$ più la u si schiaccia all'infinito più $\hat{u}$ è regolare \\ In particolare $||x||^nu\in L^1(\mathbb{R}^n) \implies \hat{u}\in C^{\infty}(\mathbb{R}^n) \ \forall n\in\mathbb{N}$\\ \\
%
Oss. Teo.4 (2) $\implies$ più u è regolare più la $\hat{u}$ si schiaccia all'infinito \\ In particolare $u\in C^{\infty}(\mathbb{R})$ t.c. $u^{(j)}\in L^1(\mathbb{R}) \ \forall j\in\mathbb{N} \implies \widehat{(u^{(j)})}(\xi)=(i\xi)^j\hat{u}(\xi)$\\
Ovvero $\widehat{(u^{(j)})}(\xi)=o(|\xi|^j)$ per $|\xi|\to +\infty$\\




\subsection{Inversione della trasformata di Fourier}
%
Oss. Trovare u data $\hat{u}$ da problemi, infatti $\mathcal{F}(L^1(\mathbb{R}^n))\not\subset L^1(\mathbb{R}^n)$\\ \\
%
%
Teo. Formula di inversione \\
Sia $u\in L^1(\mathbb{R}^n)$ t.c. \ $u\in C^0(\mathbb{R}^n)\cap L^{\infty}(\mathbb{R}^n)$ e \ $\hat{u}\in L^1(\mathbb{R}^n)$ \ allora \ $u(x)=\frac{1}{(2\pi)^n}\int_{\mathbb{R}^n}e^{ix\xi}\ \hat{u}(\xi)\ d\xi$

\begin{defi}
Data una funzione $v$, la funzione $\check{v}$ data dalla formula di inversione si dice \textbf{antitrasformata}
\end{defi}
%
Oss. In generale non vale che $\check{(\widehat{u})}=u$, servono condizioni su u %ci vorrebbe \\



\subsection{Trasformata di Fourier per distribuzioni temperate}
%
%
\begin{defi}
Sia $u\in\mathcal{S}'(\mathbb{R}^n)$, la \textbf{trasformata di Fourier} di u è il funzionale\\ 
$\hat{u}(\phi)= \ <\hat{u},\phi> \ := \ <u,\hat{\phi}> \ =\int u\hat{\phi} \ \ \ \forall \phi\in\mathcal{S}(\mathbb{R}^n)$
\end{defi}
%
Teo: $\hat{u}$ è una trasformazione temperata ($\mathcal{S}'(\mathbb{R}^n)$)
\\ \\
Dim:\\
Grazie a Fubini-Tonelli $\int_{\mathbb{R}^n}u\hat{v}=\int_{\mathbb{R}^n}\hat{u}v \ \ \ \forall u,v \in L^1(\mathbb{R}^n),$ \ \ di conseguenza vale il seguente lemma:\\
%
$\forall u\in\mathcal{S}'(\mathbb{R}^n)$ l'applicazione lineare \ $\hat{u}: v \mapsto <u,\hat{v}> \ \forall v\in\mathcal{D}(\mathbb{R}^n)$ \ \ è una distribuzione temperata\\
Quindi \ $v\in\mathcal{D}(\mathbb{R}^n)\implies\hat{v}\in\mathcal{S}(\mathbb{R}^n)$\\
%
%
Inoltre $\phi\in\mathcal{S}(\mathbb{R}^n)\implies \  \hat{\phi}\in\mathcal{S}(\mathbb{
R}^n) \ e \ \ \check{\phi}\in\mathcal{S}(\mathbb{
R}^n)$\\
$\mathcal{F}: \phi\mapsto\hat{\phi}$ per $\phi\in\mathcal{S}(\mathbb{R}^n)$ \ \ è un'applicazione biunivoca e bicontinua \\
Ovvero $\phi_n\to\phi\implies \mathcal{F}(\phi_n)\to\mathcal{F}(\phi)$, ma anche $\mathcal{F}^{-1}(\phi_n)\to\mathcal{F}^{-1}(\phi)$\\
%
$\hat{u}:\mathcal{S}(\mathbb{R}^n)\to\mathbb{R}$ è ben definita e lineare\\
Inoltre dato $\{\phi_n\}\subset\mathcal{S}(\mathbb{R}^n)$ t.c. $\phi_n\to 0$ allora essendo $\mathcal{F}$ biunivoca e bicontinua \ $\hat{\phi}_n\to 0 $ \\
Quindi $<u,\hat{\phi}_n> \ \to 0 \ \implies \hat{u}$ è continua in $\phi=0$, ma essendo lineare $\hat{u}$ è continua in $\mathcal{S}(\mathbb{R}^n)$\\
Concludiamo $\hat{u}\in\mathcal{S}'(\mathbb{R}^n)$ \\ \\
%
%
%
Oss. Si può provare che anche $\hat{u}\in\mathcal{S}'(\mathbb{R}^n)$ gode delle proprietà della trasformata classica ($\hat{u}\in L^1$)\\


\begin{defi}
Sia $u\in\mathcal{S}'(\mathbb{R}^n)$, l'\textbf{antitrasformata} $\check{u}$ di u è la distribuzione temperata \ \ $<\check{u},\phi> \ := \ <u,\check{\phi}>$
\end{defi}
\phantom{}\\
%
%
%
Teo. $\mathcal{F}:\mathcal{S}'(\mathbb{R}^n)\to\mathcal{S}'(\mathbb{R}^n)$ è lineare, biunivoca e bicontinua t.c. $\check{u}=\mathcal{F}^{-1}(u)$ \\ 
%
Oss. $\forall u\in\mathcal{S}'(\mathbb{R}^n)$ avremo $\widehat{(\check{u})}=u$\\ \\
%
%
Oss. Affinchè valga la fomula integrale per l'antitraformata basta che $u\in\mathcal{S}'(\mathbb{R}^n)$ e $\hat{u}\in L^1(\mathbb{R}^n)$\\



\subsection{Trasformata nello spazio $\mathbf{L^2}$}
%
Oss. $f\in L^p(\mathbb{R}^n) \implies u_f\in\mathcal{S}'(\mathbb{R}^n) \implies \hat{u}_f\in\mathcal{S}'(\mathbb{R}^n)$\\ \\
%
Teo. Sia $u\in\mathcal{S}'(\mathbb{R}^n)$ allora \ \ $u\in L^2(\mathbb{R}^n)\Longleftrightarrow \hat{u}\in L^2(\mathbb{R}^n)$ \\ in tal caso vale l'identità di Plancherel \ $||\hat{u}||^2_{L^2} = (2\pi)^n||u||^2_{L^2}$\\ \\
%
Cor. $\mathcal{F}:L^2(\mathbb{R}^n)\to L^2(\mathbb{R}^n)$ è biunivoca e bicontinua\\ \\
%
%
Oss. Se $u\in L^2(\mathbb{R}^n)$ non si può in generale usare la formula integrale per $\hat{u}$\\
Tuttavia se considero una successione crescente $\{K_h\}_{h\in\mathbb{N}}$ di compatti di $\mathbb{R}^n$, la cui unione è $\mathbb{R}^n$\\
Posto $u_h=\Chi_{K_h}u \ \ \in L^2(\mathbb{R}^n)\cap L^1(\mathbb{R}^n) \ \ \ \ u_h \to u \ in \ L^2(\mathbb{R}^n)$\\
Dunque $\hat{u}_h$ si calcola con l'integrale e se la successione delle trasformate converge q.o. \!\!\! allora\\ $\hat{u}(\xi)=lim_{k\to\infty}\int_{K_h}e^{-i\xi x}u(x) dx$\\ \\ \\ %ce ne vorrebbe solo uno



\subsection{Prodotto convoluzione e trasformata Fourier}
%
\begin{defi}
Siano $f,g\in L^1(\mathbb{R}^n)$, il \textbf{prodotto di convoluzione} è $(f*g)(x)=\int_{\mathbb{R}}f(x-y)g(y)dy\ \ \in L^1(\mathbb{R}^n)$
\end{defi}
%
Teo. $\widehat{(f*g)}(\xi)=\widehat{f}(\xi)\hat{g}(\xi)$\\
%
%
\begin{defi}
Siano $u\in\mathcal{D}'(\mathbb{R}^n)$ e $v\in\mathcal{D}(\mathbb{R}^n)$, il \textbf{prodotto di convoluzione tra distribuzioni e funzioni} è l'applicazione $(u*v)(x)= \ <u(\cdot),v(x-\cdot)>\ =\int u(y)v(x-y) dy $ \ \  che è una funzione $C^{\infty}(\mathbb{R}^n)$
\end{defi}
%
Oss. La definizione ha senso con l'integrale solo se u ha una funzione associata \\
Dentro l'integrale per u si intende la funzione, per questo si integra su $\mathbb{R}^n$\\ \\
%
Oss. Inoltre avremo $D^{\alpha}(u*v)=D^{\alpha}u *v=u*D^{\alpha}v$\\ \\
%
%
Oss. Nel caso $u\in\mathcal{S}'(\mathbb{R}^n), \ v\in\mathcal{S}(\mathbb{R}^n)$ \ \ $u*v \in \mathcal{S}'(\mathbb{R}^n)\cap C^{\infty}(\mathbb{R}^n)$\\ \\
%
%
Teo. Valgono le seguenti: \\
i) $u\in L^1(\mathbb{R}^n), v\in L^2(\mathbb{R}^n) \ \implies \hat{u*v}=\hat{u} \ \hat{v}$\\
ii) $u,v\in L^2(\mathbb{R}^n) \ \ \implies \hat{u*v}=\hat{u}\ \hat{v} \ \in L^2(\mathbb{R}^n)$\\
iii) $u\in\mathcal{S}'(\mathbb{R}^n), v\in\mathcal{S}(\mathbb{R}^n) \ \ \implies \hat{u*v}=\hat{u}\ \hat{v} \ \in \mathcal{S}'(\mathbb{R}^n)$\\ \\
%
%
Oss. $f,g\in L^1(\mathbb{R}^n)$ o $f,g\in L^2(\mathbb{R}^n)$ o $f\in L^1(\mathbb{R}^n), \ g\in L^2(\mathbb{R}^n) \implies$ prod convoluzione commutativo\\ \\
%
Oss. $\{u_n\}_{n\in\mathbb{N}}\subset\mathcal{S}'(\mathbb{R}^n): \ u_n\to u \ in \ \mathcal{S}'(\mathbb{R}^n)$ e $\{v_n\}_{n\in\mathbb{N}}\subset\mathcal{S}(\mathbb{R}^n): \ v_n\to v \ in \ \mathcal{S}(\mathbb{R}^n)$ allora\\
$u_n * v_n \to u*v \ in \ \mathcal{S}'(\mathbb{R}^n)$\\



\subsection{Applicazioni}
%
\phantom{}\\
\textbf{Esempi di EDO}\\ \\
%
Consideriamo una edo a coefficienti costanti di operatore L:\ \ $L u(x) := \sum_{k=0}^n a_k u^{(k)}(x)=f(x)$\\
Per risolvere l'equazione risolviamo nel caso $f=\delta$ e troviamo una soluzione $E(x)$, \ dunque $LE=\delta$. \\ 
Allora, se ha senso $E*f$, vale che $L(E*f)=(LE)*f=\delta*f=f \implies u(x)=E*f$ è una soluzione\\
E si chiama \textbf{soluzione fondamentale} dell'operatore L\\ \\
%
%
Oss. Data $-u''+u=f \ con \ f\in L^1(\mathbb{R})$ \ \ $E(x):=\frac{e^{-|x|}}{2}$ \ e \  $u(x)=\int_{\mathbb{R}}\frac{e^{-|x|}}{2}f(y)dy$\\ 
Poichè E l'abbiamo trovata con Fourier, non è l'unica soluzione\\
Ci sono anche le soluzioni dell'omogenea $v(x)=c_1e^x+c_2e^{-x}$ che non sono Fourier-trasformabili\\ \\
%
Oss. La soluzione fondamentale di $u''+u=f$ è \ \ $E(x)=\frac{sinx}{2}sign(x)$ \ \ $E*f$ ha senso se $f\in L^1$\\
Però bisogna risolvere il problema di divisione dell'omogenea (f=0) $(1-\xi^2)\hat{v}(\xi)=0$, quindi\\ l'integrale generale della edo ha la forma $U(x)=c_1sinx+c_2cosx+\int_{\mathbb{R}}E(x-y)f(y)dy$ con $c_1,c_2\in\mathbb{C}$\\ \\
%
%
Oss. Questa edo risolve i moti armonici, che è un contesto periodico, cioè con f e u periodiche\\
%
Teo. Ogni $u\in\mathcal{D}'(\mathbb{R})$ T-periodica è temperata.\\
Inoltre $\forall \{u_n\}_{n\in\mathbb{N}}\subset\mathcal{D}'(\mathbb{R})$ con $u_n$ T-periodica e convergente in $\mathcal{D}'(\mathbb{R})$, allora converge in $\mathcal{S}'(\mathbb{R})$\\ %ce ne vorrebbero 3
%
%
%
\textbf{Equazione di Poisson}\\ \\
%
Problema: Sia $\Omega$ un aperto semplicemente connesso e sia $\underline{F}:\Omega\to\mathbb{R}^n$ un campo di forze\\
Sia $\underline{F}$ irrotazionale, regolare e soggetto a una distribuzione di sorgenti $f:\Omega\to\mathbb{R}$ ovvero t.c. $-div\underline{F}=f$\\
Equivale al problema $-\Delta U=f$ in $\Omega$ \ \ questa è detta equazione di Poisson\\ \\
%
%
Oss. Per risolvere $-\nabla u = f$ dobbiamo trovare la soluzione fondamentale E(x) $\implies u(x)=(E_n * f)(x)$\\
%
Oss. Le soluzioni di $-\nabla u =\delta_0$ sono: \ $E(x)=\begin{cases}
\frac{1}{2\pi}ln\frac{1}{|x|} \ \ N=2 \\ \frac{1}{4\pi |x|} \ \ N=3
\end{cases}$ \\ \\


\end{document}